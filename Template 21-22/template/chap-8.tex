\chapter{Conclusion}
\label{cha:conclusion}
\label{con:intro}
\todo{intro conclusion}
In this chapter we will concluded the dissertation by discussing ...
In this chapter we will discuss how we build our fuzzers, what issues we had to circumvent and how we did that. 

% own results, multiple runs due toe randomness, how measured
% comparision with other fuzzers
% why did we (not) do code coverage ?
% timeout on the minizinc, but not on the ortools -> manuel work
% if stoppted before it wwas done due to timeout solvers didn't agree on the amount of solutions
% meta testing requires work



\section{STORM}
Techniques are best applied during development, since the modified storm of to often detects the already fond bugs

conclusion? 
working with seeds does limit us

%problems limited by global fucntions of used seeds (somewhat fine),
%only 'and' and 'not' combinations Just like STORM


\section{Metamorphic testing}
bug wound bound would have been found if we logged the crashes of the original file 

did not find ~(~()) simply because we dind't think to check it specificly, we did check ~=0
all checks ? + combo-able

maybe for conclusion
manual written (= work, although not much work), own choice on how complex you make the relations(simple ones work too), still need the creativity, all relations seem to be "why would this crash this is pointless to write" but do crash sometimes


\section{Future work}
% new code = new bugs
% testing configuration space aswell \cite{42FalconFuzzingConfigurationSettingsAndNormal}
% don't report already found bugs, have tests run with knowledge of the previous found bugs, not with exceptions




%\section{Conclusion}
%The final section of the chapter gives an overview of the important results
%of this chapter. This implies that the introductory chapter and the
%concluding chapter don't need a conclusion.

%%% Local Variables: 
%%% mode: latex
%%% TeX-master: "thesis"
%%% End: 

\documentclass[master=cws,masteroption=se,english]{kulemt}
\setup{% Verwijder de "%" op de volgende lijn bij UTF-8 karakterencodering
  %inputenc=utf8,
  title={The best master's thesis ever},
  author={ing. Ruben Kindt},
  promotor={Prof.\,dr.\ Tias Guns},
  assessor={ },
  assistant={Ir. Ignace Bleukx}}
% Verwijder de "%" op de volgende lijn als je de kaft wil afdrukken
%\setup{coverpageonly}
% Verwijder de "%" op de volgende lijn als je enkel de eerste pagina's wil
% afdrukken en de rest bv. via Word aanmaken.
%\setup{frontpagesonly}

% Kies de fonts voor de gewone tekst, bv. Latin Modern
\setup{font=lm}

% Hier kun je dan nog andere pakketten laden of eigen definities voorzien
\usepackage{todonotes}
\usepackage{pifont} % used for checkmarks
%\usepackage[style=numeric,sortcites,sorting=nty,backref,hyperref]{biblatex}
%\usepackage[style=numeric,sortcites,sorting=none,backref,hyperref]{biblatex} % no sort, for finding unused ref
%\nocite{*}  % cite all ref, for finding unused ref
\usepackage{biblatex}
\bibliography{references}

% Tenslotte wordt hyperref gebruikt voor pdf bestanden.
% Dit mag verwijderd worden voor de af te drukken versie.
\usepackage[pdfusetitle,colorlinks,plainpages=false]{hyperref}

%%%%%%%
% Om wat tekst te genereren wordt hier het lipsum pakket gebruikt.
% Bij een echte masterproef heb je dit natuurlijk nooit nodig!
%\IfFileExists{lipsum.sty}%
% {\usepackage{lipsum}\setlipsumdefault{11-13}}%
% {\newcommand{\lipsum}[1][11-13]{\par Hier komt wat tekst: lipsum ##1.\par}}
%%%%%%%

%\includeonly{chap-n}
\begin{document}

\begin{preface}
  I would like to thank everybody who kept me busy the last year,
  especially my promoter and my assistants. I would also like to thank the
  jury for reading the text. My sincere gratitude also goes to my wive and
  the rest of my family. \todo{replace template by real one}
\end{preface}

\listoftodos
\tableofcontents*

\begin{abstract}
	\todo{abstract}
  The \texttt{abstract} environment contains a more extensive overview of
  the work. But it should be limited to one page.
\end{abstract}

\begin{abstract*}
	\todo{samenvatting}
  In dit \texttt{abstract} environment wordt een al dan niet uitgebreide
  Nederlandse samenvatting van het werk gegeven.
  Wanneer de tekst voor een Nederlandstalige master in het Engels wordt
  geschreven, wordt hier normaal een uitgebreide samenvatting verwacht,
  bijvoorbeeld een tiental bladzijden. 
\end{abstract*}

% Een lijst van figuren en tabellen is optioneel
%\listoffigures
%\listoftables
% Bij een beperkt aantal figuren en tabellen gebruik je liever het volgende:
\listoffiguresandtables
% De lijst van symbolen is eveneens optioneel.
% Deze lijst moet wel manueel aangemaakt worden, bv. als volgt:
\chapter{List of Abbreviations and Symbols}
\section*{Abbreviations}
\begin{flushleft}
  \renewcommand{\arraystretch}{1.1}
  \begin{tabularx}{\textwidth}{@{}p{12mm}X@{}}

%    delta-debugging & "" \\
%    catigorize  & "" \\
%    grammer based fuzzing & "" \\
%    blackbox fuzzing & "" \\
%    gray box fuzzing & "" \\
%    white box fuzzing & "" \\
%    hierarchical delta-debugging & "" \\
%    1-minimal & "" \\
%    grammar based fuzzing & "" \\
%    SAT by construction & "" \\
%    UNSAT by construction & "" \\

%Alphabetic pls
	CI/CD &  Continuous Integration and Continuous Deployment, a pipeline for newly written code.to repeatably be: build, test, release, deploy and more. \\
    CP & Constrain Programming Language sometimes also referred to as CPL \\
    CPL & Constrain Programming Language also referred to as CP \\
    CSP & Constraint Satisfaction Problem is a problem with constraints and variable with a specific domain e.g. finite.\\
    CPMpy & \\
    PUT & Program Under Test, the piece of code, application of program we are testing on for potential bugs. \\
    LLVM & Although it looks like an abbreviations, it is not. LLVM is the name of a project focused on compiler and toolchain technologies. \\
    MUS & minimal unsat subset =/=  \\
    SMT & Satisfiability Modulo Theory\\
    SUT & Sofware under Test \\
  \end{tabularx}
\end{flushleft}

\section*{Symbols}
\begin{flushleft}
  \renewcommand{\arraystretch}{1.1}
  \begin{tabularx}{\textwidth}{@{}p{12mm}X@{}}
  	$\neg$ & negation \\
  	$\land$ & logical and \\
  	$\lor$  & logical or \\
  \end{tabularx}
\end{flushleft}

% Nu begint de eigenlijke tekst
\mainmatter

\chapter{Introduction}
\label{cha:intro}
Programmers make mistakes, just like everyone.
Software is often complex, written by multiple people, sometimes used with the wrong assumptions or does not meet the objective goals. 
There are a lot of causes for bugs: software complexity, multiple people writing different parts of software, changing objective goals of software, misaligned assumptions and more. Most these things can not be avoided during the creation of software but do cause program crashes, vulnerabilities, wrong outcomes and more.
Multiple forms of prevention have been created in various forms of software testers, documentation, automatic tests, code reviews. All of these aim to prevent the occurrence of bugs. While automatic test cases often evaluate goals of software end evaluate previous known bugs, it can do much more.
Fuzzing software is one of those things, a technique that is popular in the security world for exploit prevention, which generates random input for a program under test (PUT) and monitors if the program crashes or not. This explanation was the original interpretation of fuzzing as preformed by \cite{originalFuzzingUnixUtils}


The first contains a general introduction to the work. The goals are\todo{remove this} defined and the modus operandi is explained.

%%% Local Variables: 
%%% mode: latex
%%% TeX-master: "thesis"
%%% End: 

\chapter{Fuzzing}
\label{cha:2:fuzzing}
A chapter is a logical unit. It normally starts with an introduction, which
you are reading now. The last topic of the chapter holds the conclusion.
\todo{intro chapter 1}

%if inf seed set, reducing it is better with the following technique (the min seed subset without any code coverage loss) \cite{14rebert2014seedselecting}
%seed selection problem, inf seeds which one do we use
%other fuzzers internal workings
%Chapter around fuzzing
%	\cite{mathis2019parser} states 3 optins traditional, stochastic and syntax driven.
%avoiding inf loops -> timeouts
%
%subsection about (psuedo-)randomness


\section{History} \todo{fix title}
The rise of fuzzing came when Miller gave a classroom assignment\cite{21FuzzingAssignment} in 1988 to his computer science students to test Unix utilities with randomly generated inputs with the goal to break the utilities. Two years later in December he wrote a paper\cite{4originalFuzzingUnixUtils} about the remarkable results. That more than 24\% to 33\% of the programs crashed.
In the last thirty years the technique of fuzzing has changed significantly and various classifications have come forward\cite{12Fuzzingasurvey}\cite{13manes2019survey}. Three most used classifications are: 
how does the fuzzer create input, how well is the input structured and does the fuzzer have knowledge of the program under test (PUT)?


%generation or mutatuion
%black, gray white box
%lexical, sematic, constraint, random

\section{Generation of mutation}
A fuzzer can construct input for a PUT in two ways, it can generate input itself or it can take an existing input and modify it, which are often called seeds. While Generation is more common than modifying when it comes to in smaller inputs the opposite is true for larger inputs. This is cause by the fact that generating semi-valid input becomes a lot harder the longer the input becomes. For example generating the word "Fuzzing" by uniformly random sampling using ASCII, has a chance of one in $5*10^{14}$ of happening, making this technique infeasible when we want to generate bigger semi-valid inputs. With mutation we can start of with larger and valid input and make modifications to create semi-valid inputs. With this last technique the diversity of the seeding inputs does become quite important. Ideally we would have an unlimited diverse set of inputs, but due to limited computation and available inputs we sometimes need to take a subset. In a paper by Rebert et al. \cite{14rebert2014seedselecting} they say that seed selection algorithms can help and compare random seed selection to the minimal subset of seeds with the highest code coverage among other seed selection algorithms. 

\section{Input structure}
%lexical, sementical, constraint or random
While we have discussed the bigger scope on how inputs are created, let us go into more detail. As we have seen before fuzzing started mainly with Miller's assignment, that random generation of inputs falls under 'dumb' fuzzing. Due to only seeing the input as one long list of strings with no knowledge of any substrings. This technique can be applied to mutation as well, compared to only adding with generation here we also additionally remove or change randomly selected symbols of the seed with new symbols. 
We can create three types of inputs: valid, semi-valid and nonsense input. With nonsense input we will be testing the parser almost exclusively, either it crashes the parser or the parser will say invalid and the PUT will stop running. With semi-valid input we hope to be as close as possible to valid input to be able to explore the PUT but to catch an edge cases and crash.

A smarter technique is referred to one which has knowledge about the structure inputs can have or should have. This increases the chance of inputs passing the parser, being able to test the deeper parts of the PUT and as such covering more of the PUT. This with an increased complex fuzzer. We can build a 'smart' fuzzer by adding knowledge about keywords, making it a lexical fuzzer, adding knowledge about syntax, for example all parentheses needs to be matched. Directed fuzz testing does fit in this category as well but is only possible in a white box environment, more on that later.

\section{Black, grey or white box fuzzing}
Now that we have discussed adding knowledge of inputs to the fuzzer, we can also add knowledge about the PUT to the fuzzer. Which brings us to black, grey and white box fuzzing. With black box fuzzing we have no knowledge about the working of the PUT and we treat the PUT as a literal black box, we present out input and we look at what comes out of it. With only minimal information the fuzzer then tries to improve its input creation. This was also the technique that Miller (unknowingly) used.
 
With grey box fuzzing is usually comes with tools that give indirect information to the fuzzer, tools like: code coverage measurements, timings, types of errors and more.

Then we arrived at white box testing, with this technique fuzzers can know the source code and can adjust there inputs to fuzz specific parts of the code at a higher cost due to having to reverse engineer the path to specific edge cases. You may may have already suspected, white box fuzzing has more knowledge and can find more bugs per input but creating those inputs take more time compared to black box fuzzing.

The differentiation between black, grey and white box fuzzing is not clear cut, most people would agree that white box fuzzing has full knowledge about the PUT, including the source code, that grey box fuzzing has some knowledge about the PUT and that black box fuzzing has little to no knowledge about the PUT. Going into more detail all we can say is that it is no longer a black-and-white situation and that the lines become more fuzzy. \todo{is this pun allowed?}


\section{Fuzzing small programs} \todo{fix title}
\section{Fuzzing big programs}

\subsection{clasical fuzzing (for some original cite's)}
\subsection{AFL++}
\subsection{KLEE}
\subsection{ClusterFuzz}
\subsection{STORM (startingpaper)}
A novel way of approaching this oracle problem is by Alexandra Bugariu and Peter M\"uller in "Automatically testing string solvers"\cite{9bugariu2020automaticallyTestingStringSolvers} where they know the (un)satisfiability of the formulas by the way of construction, called (un)sat by construction.  For formulas that satisfy they generate trivial satisfiable formulas and then by satisfiability preserving transformations create more complex formulas. And unsatisfiable formulas they use $\neg$ A $\land$ A', with A' being a equivalent formula of A, to create the trivial unsatisfiable formulas. To increase the complexity of those trivial formulas, they again depend on satisfiability preserving transformation.
This technique has also been applied to SMT solvers by Muhammad Numair Mansur at al.\cite{1mansur2020detecting}

\subsection{types of bugs} \cite{1mansur2020detecting}

\subsection{oracle (what is a crash)}
%oracle problem? see holy grail
%somewhere a ref to later chapter input simplification (minimisation differantioation)

\section{Conclusion}
The final section of the chapter gives an overview of the important results
of this chapter. This implies that the introductory chapter and the
concluding chapter don't need a conclusion.


%%% Local Variables: 
%%% mode: latex
%%% TeX-master: "thesis"
%%% End: 

\chapter{Detecting crucial parts of the input Simplifying Inputs} \todo{better title}
\label{cha:3}
When we detect that the PUT crashes, wrongly satisfies, wrongly not satisfies or hangs on a given input we now want to know why it does that. What causes this unwanted output and what line the bug occurs. With crashes, a stack trace and some luck this could be easy, but when the crash is not main perpetrator or we get an other unwanted output the developer my need to debug deep into the code to find the bug. This with a potential large input could be a tedious and long assignment, for this reason we would like to know what parts of the input are related to the bug. We will discover this further in this chapter, starting with \todo{plug in sections}

\section{Minimizing inputs}
As said with bigger inputs it takes longer to find the bugs due to having to find identify which parts of the input trigger the error or having to walk through the execution of the PUT if even possible. This is where techniques come into play to create detect the crucial parts of the input. This while still finding the same bug and not changing a null pointer dereference to a parser related bug for example.\cite{bookZellerwhyProgramsFail}

\subsection{Simplifying}
The first and the most used technique is the simplification of the inputs where we remove parts of a failing input and check if it still fails. When no longer possible to remove any part of the input we have obtained an input where all parts are needed to expose the bug. This input is at the same time also the shortest possible input to trigger this bug, making finding the bug easier than in the original input filled with unrelated parts. 
\todo{add mozilla ref}
\subsection{Isolation}
Another technique, isolation, is explained in Andreas Zeller et al. 
\cite{5zeller2002simplifyingIsolatingFailure-inducing} this is a technique where instead of minimizing the input we try to find the smallest difference between an input that shows the bug versus an input that does not show it. This with the advantage that no matter the if we find the bug or not the difference will diminish, either the maximum input will shrink or the minimum input will grow. Although we will have to make sure that we are still finding the same bug and not moving from a null pointer dereference to a parser related bug. Isolation also brings extra complexity with the tracking of multiple inputs and the maximal input could take longer to run due to its size, but according to Andreas Zeller et al. is the faster one .

% what size to remove, importent for speed
% what are both, what situations is which the better one?  donno atm
%	Isolation will require a lot more things to track but is faster
\cite{bookZellerwhyProgramsFail}
%	isolation vs simplifying \cite{zeller2009programs} p 285
%
%The Precision Effect minimizing/simplyfying may lead to a dif found bug if multiple, since all need to be solved not problem  \cite{zeller2002simplifying}
%	 can be solved with stack trace comparison
\subsection{Small inputs by creation}
small by construction source 9

\todo{algorithms?} paper 5 has nice simplification to isolation delta deb, called 'ddmin()'

\section{Delta-debugging}
\cite{2FuzzingAndDelta-debuggingSMTSolvers}


%chapter on simplifying the crashes
%	binary search will not work all the time
%	quarters remove may work (if all parts fail go more granular, 1/9 or 1/16)
%		start with halfs then *2
%		always search further with same granularity but with removed part until all options with that granularity searched \cite{zeller2009programs} p111
%		this uses no knowledge from input structure and program structure \cite{zeller2009programs} p112
%	delta debugging
%		time spend searching vs simplified ratio is important as mentioned in \cite{mansur2020detecting}
%		and needs to preserve satisfiability as mentioned in \cite{mansur2020detecting}
%		^ possibly a big deal to find critical bugs
%
%	with knowledge of input, syntax \cite{zeller2009programs}
%	of by bigger entities like lines of words \cite{zeller2009programs}
%	 for speed
%
%	alt approach like \cite{mansur2020detecting} try finding the bug again with less resources avail
%	or isolaytion \cite{zeller2009programs} p 285 
%		I think it may fail if multiple parts are relevant
%		I think it could detect for example the CPMpy import as a bug cause as the min diff that causes the bug
%
%	sub section on MUS/minimum unsat subset vs delta debugging
%		MUs good for only whole constraints while 
%		delta debugging goes for partial structures
%

\section{Deduplication}
Another thing to notice is that multiple inputs could prompt the same bug from occurring, these inputs could be similar but don't have to be. With simplifying the input we should be able to detect exact copies, but depending on the simplification's time complexity other techniques could be better with similar results. In case where we would have access to stack traces (via crashes or hanging PUT's) we could differentiate the bugs on basis of the hash of multiple lines from the backtrace sometimes even numerous hashes per input. this technique is called stack backtrace hashing and is quite popular according to Valentin J.M. Man\`es et al.\cite{13manes2019survey}. Another technique talked about in that paper, is looking at the code coverage generated by the inputs where we use the executed path (or hash of it) is used as a fingerprint of the inputs. A technique, used by Microsoft\cite{36semanticsAwareDeduplicationRETracer} is called semantics based deduplication, where in stead of back track use memory dumps to hopefully find the origins of bugs. This use of dumps is less ideal due to traces having more information, but the latter is not always possible due to the performance overhead and privacy causes as specified in the paper. A last technique would be looking at the bug description left by a manual bug reports by the user, although this dependence on the quality of the bug reports and is most likely poorly automatable. None of the techniques mentioned above are perfect: with stack backtrace hashing you could find to many false positives or false negatives depending on the depth taken from the stack, with coverage some inputs will generate extra function calls and the semantics based deduplication are limited to X86 or x86-64 code with the binary file and the debug information. Neither of these techniques work with black box fuzzing unfortunately.

\section{Conclusion}
The final section of the chapter gives an overview of the important results
of this chapter. This implies that the introductory chapter and the
concluding chapter don't need a conclusion.


%%% Local Variables: 
%%% mode: latex
%%% TeX-master: "thesis"
%%% End: 

\chapter{CCPMpy}
\label{cha:4}
\todo{intro ch 4}

\section{The First Topic of this Chapter}

% see holy grail
%
%\section{Figures}
%Figures are used to add illustrations to the text. The \fref{fig:logo} shows
%the KU~Leuven logo as an illustration.
%\begin{figure}
%	\centering
%	\includegraphics{logokul}
%	\caption{The KU~Leuven logo.}
%	\label{fig:logo}
%\end{figure}
%
%\section{Tables}
%Tables are used to present data neatly arranged. A table is normally
%not a spreadsheet! Compare \tref{tab:wrong} en \tref{tab:ok}: which table do
%you prefer?
%
%\begin{table}
%	\centering
%	\begin{tabular}{||l|lr||} \hline
%		gnats     & gram      & \$13.65 \\ \cline{2-3}
%		& each      & .01 \\ \hline
%		gnu       & stuffed   & 92.50 \\ \cline{1-1} \cline{3-3}
%		emu       &           & 33.33 \\ \hline
%		armadillo & frozen    & 8.99 \\ \hline
%	\end{tabular}
%	\caption{A table with the wrong layout.}
%	\label{tab:wrong}
%\end{table}
%
%\begin{table}
%	\centering
%	\begin{tabular}{@{}llr@{}} \toprule
%		\multicolumn{2}{c}{Item} \\ \cmidrule(r){1-2}
%		Animal    & Description & Price (\$)\\ \midrule
%		Gnat      & per gram    & 13.65 \\
%		& each        & 0.01 \\
%		Gnu       & stuffed     & 92.50 \\
%		Emu       & stuffed     & 33.33 \\
%		Armadillo & frozen      & 8.99 \\ \bottomrule
%	\end{tabular}
%	\caption{A table with the correct layout.}
%	\label{tab:ok}
%\end{table}


\section{Conclusion}
The final section of the chapter gives an overview of the important results
of this chapter. This implies that the introductory chapter and the
concluding chapter don't need a conclusion.

%%% Local Variables: 
%%% mode: latex
%%% TeX-master: "thesis"
%%% End: 

%\chapter{Evaluation}
\label{cha:eval}
\label{cha:intro}
\todo{eval:intro}

% own results, multiple runs due to randomness, how measured
% comparision with other fuzzers

\section{Research questions}
\label{eval:RQ}
In this chapter we will try to formulate answers to the following research questions.
\subsection{main focus: fuzzing technique-focused}
RQ1: What fuzzing technique will find the most bugs?\\
RQ2: What fuzzing technique will find the most critical bugs?\\
RQ3: What type of bugs will be found using which fuzzing technique?\\
\subsubsection{subfocuses}
RQ4: Which metamorphic transformation find the most (critical) bugs?\\
focused on: \\
bool, int, list, array
\subsection{solver-focused}
RQ5: Which solver has the most (critical) bugs?\\
\subsection{classification-focused}
RQ6: How many (critical) bugs can we find?\\
RQ7: What the causes of the solvers\\
RQ8: What are the type of bugs found?\\
\subsection{not our focus: efficient-focused} 
RQ9: How efficient is the fuzzer?\\
RQ10: How efficient is the deobfuscator?\\
RQ11: How well does the deobfuscator deobfuscate the inputs?\\

%challenges tegengekomen en oplossingen
pySSD: graph maker

\section{Conclusion}
\label{eval:conclusion}
\todo{conclusion}

%%% Local Variables: 
%%% mode: latex
%%% TeX-master: "thesis"
%%% End: 

%\chapter{Approach}
\label{cha:v}
\todo{intro ch v}

\section{STORM}
\section{Differential testing}
\subsection{STORM}
\subsection{YinYang}

\section{Seeds}
\footnote{\url{https://github.com/hakank/hakank/tree/master/cpmpy}} and \cite{18bleukx2022model}
\footnote{\url{https://github.com/CPMpy/cpmpy/tree/master/examples}}

\section{Conclusion}
\todo{Conclusion}
The final section of the chapter gives an overview of the important results
of this chapter. This implies that the introductory chapter and the
concluding chapter don't need a conclusion.

%%% Local Variables: 
%%% mode: latex
%%% TeX-master: "thesis"
%%% End: 

%\chapter{creation of fuzzer}
\label{cha:x}
\todo{intro ch x}

% own results, multiple runs due toe randomness, how measured
% comparision with other fuzzers

\section{creation of sat and unsat formulas}
see paper 43 p 4
Semantic Fusion
\cite{43YinYang}
een ref to chapter 2

seed files came from the CPMpy repository the main branch and the csplib branch aswell from Hakank's repository downloaded on Tuesday 27/09/2022
Storm downloaded on Tuesday 27/09/2022 from https://github.com/Practical-Formal-Methods/storm 
https://github.com/Practical-Formal-Methods/storm/commit/55d091624523a0544112ffc339fe81103b3daa2b



Storm take in SMT-lib seed files, we can convert them to minizinc and then using fzn2smt but It hasn't been maintained in over a decenta and doing multiple convertions only back could be tricky and would introduce multiple layers which can introduce bugs a normal user would ever see
Therefore we will be refactoring STORM to fit our CPMpy


























%
%\section{Figures}
%Figures are used to add illustrations to the text. The \fref{fig:logo} shows
%the KU~Leuven logo as an illustration.
%\begin{figure}
%	\centering
%	\includegraphics{logokul}
%	\caption{The KU~Leuven logo.}
%	\label{fig:logo}
%\end{figure}
%
%\section{Tables}
%Tables are used to present data neatly arranged. A table is normally
%not a spreadsheet! Compare \tref{tab:wrong} en \tref{tab:ok}: which table do
%you prefer?
%
%\begin{table}
%	\centering
%	\begin{tabular}{||l|lr||} \hline
%		gnats     & gram      & \$13.65 \\ \cline{2-3}
%		& each      & .01 \\ \hline
%		gnu       & stuffed   & 92.50 \\ \cline{1-1} \cline{3-3}
%		emu       &           & 33.33 \\ \hline
%		armadillo & frozen    & 8.99 \\ \hline
%	\end{tabular}
%	\caption{A table with the wrong layout.}
%	\label{tab:wrong}
%\end{table}
%
%\begin{table}
%	\centering
%	\begin{tabular}{@{}llr@{}} \toprule
%		\multicolumn{2}{c}{Item} \\ \cmidrule(r){1-2}
%		Animal    & Description & Price (\$)\\ \midrule
%		Gnat      & per gram    & 13.65 \\
%		& each        & 0.01 \\
%		Gnu       & stuffed     & 92.50 \\
%		Emu       & stuffed     & 33.33 \\
%		Armadillo & frozen      & 8.99 \\ \bottomrule
%	\end{tabular}
%	\caption{A table with the correct layout.}
%	\label{tab:ok}
%\end{table}
%

\section{Conclusion}
The final section of the chapter gives an overview of the important results
of this chapter. This implies that the introductory chapter and the
concluding chapter don't need a conclusion.

%%% Local Variables: 
%%% mode: latex
%%% TeX-master: "thesis"
%%% End: 

% ... en zo verder tot
\chapter{The Final Chapter}
\label{cha:n}

\section{Conclusion}

%%% Local Variables: 
%%% mode: latex
%%% TeX-master: "thesis"
%%% End: 

\include{conclusion}

% Indien er bijlagen zijn:
\appendixpage*          % indien gewenst
\appendix
%\chapter{Manual for setting up the fuzzer}
\label{app:A}


%%% Local Variables: 
%%% mode: latex
%%% TeX-master: "thesis"
%%% End: 


\backmatter
% Na de bijlagen plaatst men nog de bibliografie.
% Je kan de  standaard "abbrv" bibliografiestijl vervangen door een andere.
%\bibliographystyle{abbrv}
%\bibliography{references}
\printbibliography
\end{document}

%%% Local Variables: 
%%% mode: latex
%%% TeX-master: t
%%% End: 

\documentclass[master=cws,masteroption=se,english]{kulemt}
\setup{% Verwijder de "%" op de volgende lijn bij UTF-8 karakterencodering
  %inputenc=utf8,
  title={The best master's thesis ever},
  author={ing. Ruben Kindt},
  promotor={Prof.\,dr.\ Tias Guns},
  assessor={ },
  assistant={}}
% Verwijder de "%" op de volgende lijn als je de kaft wil afdrukken
%\setup{coverpageonly}
% Verwijder de "%" op de volgende lijn als je enkel de eerste pagina's wil
% afdrukken en de rest bv. via Word aanmaken.
%\setup{frontpagesonly}

% Kies de fonts voor de gewone tekst, bv. Latin Modern
\setup{font=lm}

% Hier kun je dan nog andere pakketten laden of eigen definities voorzien
\usepackage{todonotes}
%\usepackage[style=numeric,sortcites,sorting=nty,backref,hyperref]{biblatex}
\usepackage{biblatex}
\bibliography{references}

% Tenslotte wordt hyperref gebruikt voor pdf bestanden.
% Dit mag verwijderd worden voor de af te drukken versie.
\usepackage[pdfusetitle,colorlinks,plainpages=false]{hyperref}

%%%%%%%
% Om wat tekst te genereren wordt hier het lipsum pakket gebruikt.
% Bij een echte masterproef heb je dit natuurlijk nooit nodig!
%\IfFileExists{lipsum.sty}%
% {\usepackage{lipsum}\setlipsumdefault{11-13}}%
% {\newcommand{\lipsum}[1][11-13]{\par Hier komt wat tekst: lipsum ##1.\par}}
%%%%%%%

%\includeonly{chap-n}
\begin{document}

\begin{preface}
  I would like to thank everybody who kept me busy the last year,
  especially my promoter and my assistants. I would also like to thank the
  jury for reading the text. My sincere gratitude also goes to my wive and
  the rest of my family.
\end{preface}

\listoftodos
\tableofcontents*

\begin{abstract}
	\todo{abstract}
  The \texttt{abstract} environment contains a more extensive overview of
  the work. But it should be limited to one page.
\end{abstract}

\begin{abstract*}
	\todo{samenvatting}
  In dit \texttt{abstract} environment wordt een al dan niet uitgebreide
  Nederlandse samenvatting van het werk gegeven.
  Wanneer de tekst voor een Nederlandstalige master in het Engels wordt
  geschreven, wordt hier normaal een uitgebreide samenvatting verwacht,
  bijvoorbeeld een tiental bladzijden. 
\end{abstract*}

% Een lijst van figuren en tabellen is optioneel
%\listoffigures
%\listoftables
% Bij een beperkt aantal figuren en tabellen gebruik je liever het volgende:
\listoffiguresandtables
% De lijst van symbolen is eveneens optioneel.
% Deze lijst moet wel manueel aangemaakt worden, bv. als volgt:
\chapter{List of Abbreviations and Symbols}
\section*{Abbreviations}
\begin{flushleft}
  \renewcommand{\arraystretch}{1.1}
  \begin{tabularx}{\textwidth}{@{}p{12mm}X@{}}
%    LoG   & Laplacian-of-Gaussian \\
%    MSE   & Mean Square error \\
%    PSNR  & Peak Signal-to-Noise ratio \\
%    delta-debugging & "" \\
%    catigorize  & "" \\
%    grammer based fuzzing & "" \\
%    blackbox fuzzing & "" \\
%    gray box fuzzing & "" \\
%    white box fuzzing & "" \\
%    hierarchical delta-debugging & "" \\
%    1-minimal & "" \\
%    grammar based fuzzing & "" \\
%    SAT by construction & "" \\
%    UNSAT by construction & "" \\
    MUS & minimal unsat subset =/=  \\
    PUT & Program Under Test, the piece of code, application of program we are testing on for potential bugs. \\
    CP & Constrain Programming Language sometimes also referred to as CPL \\
    CPMpy & \\
    SMT & Satisfiability Modulo Theory\\
  \end{tabularx}
\end{flushleft}

\section*{Symbols}
\begin{flushleft}
  \renewcommand{\arraystretch}{1.1}
  \begin{tabularx}{\textwidth}{@{}p{12mm}X@{}}
  	$\neg$ & negation \\
  	$\land$ & logical and \\
  	$\lor$  & logical or \\
  \end{tabularx}
\end{flushleft}

% Nu begint de eigenlijke tekst
\mainmatter

\chapter{Introduction}
\label{cha:intro}
Programmers make mistakes, just like everyone.
Software is often complex, written by multiple people, sometimes used with the wrong assumptions or does not meet the objective goals. 
There are a lot of causes for bugs: software complexity, multiple people writing different parts of software, changing objective goals of software, misaligned assumptions and more. Most these things can not be avoided during the creation of software but do cause program crashes, vulnerabilities, wrong outcomes and more.
Multiple forms of prevention have been created in various forms of software testers, documentation, automatic tests, code reviews. All of these aim to prevent the occurrence of bugs. While automatic test cases often evaluate goals of software end evaluate previous known bugs, it can do much more.
Fuzzing software is one of those things, a technique that is popular in the security world for exploit prevention, which generates random input for a program under test (PUT) and monitors if the program crashes or not. This explanation was the original interpretation of fuzzing as preformed by \cite{originalFuzzingUnixUtils}


The first contains a general introduction to the work. The goals are\todo{remove this} defined and the modus operandi is explained.

%%% Local Variables: 
%%% mode: latex
%%% TeX-master: "thesis"
%%% End: 

\chapter{Fuzzing}
\label{cha:2:fuzzing}
A chapter is a logical unit. It normally starts with an introduction, which
you are reading now. The last topic of the chapter holds the conclusion.
\todo{intro chapter 1}

%if inf seed set, reducing it is better with the following technique (the min seed subset without any code coverage loss) \cite{14rebert2014seedselecting}
%seed selection problem, inf seeds which one do we use
%other fuzzers internal workings
%Chapter around fuzzing
%	\cite{mathis2019parser} states 3 optins traditional, stochastic and syntax driven.
%avoiding inf loops -> timeouts
%
%subsection about (psuedo-)randomness


\section{History} \todo{fix title}
The rise of fuzzing came when Miller gave a classroom assignment\cite{21FuzzingAssignment} in 1988 to his computer science students to test Unix utilities with randomly generated inputs with the goal to break the utilities. Two years later in December he wrote a paper\cite{4originalFuzzingUnixUtils} about the remarkable results. That more than 24\% to 33\% of the programs crashed.
In the last thirty years the technique of fuzzing has changed significantly and various classifications have come forward\cite{12Fuzzingasurvey}\cite{13manes2019survey}. Three most used classifications are: 
how does the fuzzer create input, how well is the input structured and does the fuzzer have knowledge of the program under test (PUT)?


%generation or mutatuion
%black, gray white box
%lexical, sematic, constraint, random

\section{Generation of mutation}
A fuzzer can construct input for a PUT in two ways, it can generate input itself or it can take an existing input and modify it, which are often called seeds. While Generation is more common than modifying when it comes to in smaller inputs the opposite is true for larger inputs. This is cause by the fact that generating semi-valid input becomes a lot harder the longer the input becomes. For example generating the word "Fuzzing" by uniformly random sampling using ASCII, has a chance of one in $5*10^{14}$ of happening, making this technique infeasible when we want to generate bigger semi-valid inputs. With mutation we can start of with larger and valid input and make modifications to create semi-valid inputs. With this last technique the diversity of the seeding inputs does become quite important. Ideally we would have an unlimited diverse set of inputs, but due to limited computation and available inputs we sometimes need to take a subset. In a paper by Rebert et al. \cite{14rebert2014seedselecting} they say that seed selection algorithms can help and compare random seed selection to the minimal subset of seeds with the highest code coverage among other seed selection algorithms. 

\section{Input structure}
%lexical, sementical, constraint or random
While we have discussed the bigger scope on how inputs are created, let us go into more detail. As we have seen before fuzzing started mainly with Miller's assignment, that random generation of inputs falls under 'dumb' fuzzing. Due to only seeing the input as one long list of strings with no knowledge of any substrings. This technique can be applied to mutation as well, compared to only adding with generation here we also additionally remove or change randomly selected symbols of the seed with new symbols. 
We can create three types of inputs: valid, semi-valid and nonsense input. With nonsense input we will be testing the parser almost exclusively, either it crashes the parser or the parser will say invalid and the PUT will stop running. With semi-valid input we hope to be as close as possible to valid input to be able to explore the PUT but to catch an edge cases and crash.

A smarter technique is referred to one which has knowledge about the structure inputs can have or should have. This increases the chance of inputs passing the parser, being able to test the deeper parts of the PUT and as such covering more of the PUT. This with an increased complex fuzzer. We can build a 'smart' fuzzer by adding knowledge about keywords, making it a lexical fuzzer, adding knowledge about syntax, for example all parentheses needs to be matched. Directed fuzz testing does fit in this category as well but is only possible in a white box environment, more on that later.

\section{Black, grey or white box fuzzing}
Now that we have discussed adding knowledge of inputs to the fuzzer, we can also add knowledge about the PUT to the fuzzer. Which brings us to black, grey and white box fuzzing. With black box fuzzing we have no knowledge about the working of the PUT and we treat the PUT as a literal black box, we present out input and we look at what comes out of it. With only minimal information the fuzzer then tries to improve its input creation. This was also the technique that Miller (unknowingly) used.
 
With grey box fuzzing is usually comes with tools that give indirect information to the fuzzer, tools like: code coverage measurements, timings, types of errors and more.

Then we arrived at white box testing, with this technique fuzzers can know the source code and can adjust there inputs to fuzz specific parts of the code at a higher cost due to having to reverse engineer the path to specific edge cases. You may may have already suspected, white box fuzzing has more knowledge and can find more bugs per input but creating those inputs take more time compared to black box fuzzing.

The differentiation between black, grey and white box fuzzing is not clear cut, most people would agree that white box fuzzing has full knowledge about the PUT, including the source code, that grey box fuzzing has some knowledge about the PUT and that black box fuzzing has little to no knowledge about the PUT. Going into more detail all we can say is that it is no longer a black-and-white situation and that the lines become more fuzzy. \todo{is this pun allowed?}


\section{Fuzzing small programs} \todo{fix title}
\section{Fuzzing big programs}

\subsection{clasical fuzzing (for some original cite's)}
\subsection{AFL++}
\subsection{KLEE}
\subsection{ClusterFuzz}
\subsection{STORM (startingpaper)}
A novel way of approaching this oracle problem is by Alexandra Bugariu and Peter M\"uller in "Automatically testing string solvers"\cite{9bugariu2020automaticallyTestingStringSolvers} where they know the (un)satisfiability of the formulas by the way of construction, called (un)sat by construction.  For formulas that satisfy they generate trivial satisfiable formulas and then by satisfiability preserving transformations create more complex formulas. And unsatisfiable formulas they use $\neg$ A $\land$ A', with A' being a equivalent formula of A, to create the trivial unsatisfiable formulas. To increase the complexity of those trivial formulas, they again depend on satisfiability preserving transformation.
This technique has also been applied to SMT solvers by Muhammad Numair Mansur at al.\cite{1mansur2020detecting}

\subsection{types of bugs} \cite{1mansur2020detecting}

\subsection{oracle (what is a crash)}
%oracle problem? see holy grail
%somewhere a ref to later chapter input simplification (minimisation differantioation)

\section{Conclusion}
The final section of the chapter gives an overview of the important results
of this chapter. This implies that the introductory chapter and the
concluding chapter don't need a conclusion.


%%% Local Variables: 
%%% mode: latex
%%% TeX-master: "thesis"
%%% End: 

\chapter{Detecting crucial parts of the input Simplifying Inputs} \todo{better title}
\label{cha:3}
When we detect that the PUT crashes, wrongly satisfies, wrongly not satisfies or hangs on a given input we now want to know why it does that. What causes this unwanted output and what line the bug occurs. With crashes, a stack trace and some luck this could be easy, but when the crash is not main perpetrator or we get an other unwanted output the developer my need to debug deep into the code to find the bug. This with a potential large input could be a tedious and long assignment, for this reason we would like to know what parts of the input are related to the bug. We will discover this further in this chapter, starting with \todo{plug in sections}

\section{Minimizing inputs}
As said with bigger inputs it takes longer to find the bugs due to having to find identify which parts of the input trigger the error or having to walk through the execution of the PUT if even possible. This is where techniques come into play to create detect the crucial parts of the input. This while still finding the same bug and not changing a null pointer dereference to a parser related bug for example.\cite{bookZellerwhyProgramsFail}

\subsection{Simplifying}
The first and the most used technique is the simplification of the inputs where we remove parts of a failing input and check if it still fails. When no longer possible to remove any part of the input we have obtained an input where all parts are needed to expose the bug. This input is at the same time also the shortest possible input to trigger this bug, making finding the bug easier than in the original input filled with unrelated parts. 
\todo{add mozilla ref}
\subsection{Isolation}
Another technique, isolation, is explained in Andreas Zeller et al. 
\cite{5zeller2002simplifyingIsolatingFailure-inducing} this is a technique where instead of minimizing the input we try to find the smallest difference between an input that shows the bug versus an input that does not show it. This with the advantage that no matter the if we find the bug or not the difference will diminish, either the maximum input will shrink or the minimum input will grow. Although we will have to make sure that we are still finding the same bug and not moving from a null pointer dereference to a parser related bug. Isolation also brings extra complexity with the tracking of multiple inputs and the maximal input could take longer to run due to its size, but according to Andreas Zeller et al. is the faster one .

% what size to remove, importent for speed
% what are both, what situations is which the better one?  donno atm
%	Isolation will require a lot more things to track but is faster
\cite{bookZellerwhyProgramsFail}
%	isolation vs simplifying \cite{zeller2009programs} p 285
%
%The Precision Effect minimizing/simplyfying may lead to a dif found bug if multiple, since all need to be solved not problem  \cite{zeller2002simplifying}
%	 can be solved with stack trace comparison
\subsection{Small inputs by creation}
small by construction source 9

\todo{algorithms?} paper 5 has nice simplification to isolation delta deb, called 'ddmin()'

\section{Delta-debugging}
\cite{2FuzzingAndDelta-debuggingSMTSolvers}


%chapter on simplifying the crashes
%	binary search will not work all the time
%	quarters remove may work (if all parts fail go more granular, 1/9 or 1/16)
%		start with halfs then *2
%		always search further with same granularity but with removed part until all options with that granularity searched \cite{zeller2009programs} p111
%		this uses no knowledge from input structure and program structure \cite{zeller2009programs} p112
%	delta debugging
%		time spend searching vs simplified ratio is important as mentioned in \cite{mansur2020detecting}
%		and needs to preserve satisfiability as mentioned in \cite{mansur2020detecting}
%		^ possibly a big deal to find critical bugs
%
%	with knowledge of input, syntax \cite{zeller2009programs}
%	of by bigger entities like lines of words \cite{zeller2009programs}
%	 for speed
%
%	alt approach like \cite{mansur2020detecting} try finding the bug again with less resources avail
%	or isolaytion \cite{zeller2009programs} p 285 
%		I think it may fail if multiple parts are relevant
%		I think it could detect for example the CPMpy import as a bug cause as the min diff that causes the bug
%
%	sub section on MUS/minimum unsat subset vs delta debugging
%		MUs good for only whole constraints while 
%		delta debugging goes for partial structures
%

\section{Deduplication}
Another thing to notice is that multiple inputs could prompt the same bug from occurring, these inputs could be similar but don't have to be. With simplifying the input we should be able to detect exact copies, but depending on the simplification's time complexity other techniques could be better with similar results. In case where we would have access to stack traces (via crashes or hanging PUT's) we could differentiate the bugs on basis of the hash of multiple lines from the backtrace sometimes even numerous hashes per input. this technique is called stack backtrace hashing and is quite popular according to Valentin J.M. Man\`es et al.\cite{13manes2019survey}. Another technique talked about in that paper, is looking at the code coverage generated by the inputs where we use the executed path (or hash of it) is used as a fingerprint of the inputs. A technique, used by Microsoft\cite{36semanticsAwareDeduplicationRETracer} is called semantics based deduplication, where in stead of back track use memory dumps to hopefully find the origins of bugs. This use of dumps is less ideal due to traces having more information, but the latter is not always possible due to the performance overhead and privacy causes as specified in the paper. A last technique would be looking at the bug description left by a manual bug reports by the user, although this dependence on the quality of the bug reports and is most likely poorly automatable. None of the techniques mentioned above are perfect: with stack backtrace hashing you could find to many false positives or false negatives depending on the depth taken from the stack, with coverage some inputs will generate extra function calls and the semantics based deduplication are limited to X86 or x86-64 code with the binary file and the debug information. Neither of these techniques work with black box fuzzing unfortunately.

\section{Conclusion}
The final section of the chapter gives an overview of the important results
of this chapter. This implies that the introductory chapter and the
concluding chapter don't need a conclusion.


%%% Local Variables: 
%%% mode: latex
%%% TeX-master: "thesis"
%%% End: 

\chapter{CCPMpy}
\label{cha:4}
\todo{intro ch 4}

\section{The First Topic of this Chapter}

% see holy grail
%
%\section{Figures}
%Figures are used to add illustrations to the text. The \fref{fig:logo} shows
%the KU~Leuven logo as an illustration.
%\begin{figure}
%	\centering
%	\includegraphics{logokul}
%	\caption{The KU~Leuven logo.}
%	\label{fig:logo}
%\end{figure}
%
%\section{Tables}
%Tables are used to present data neatly arranged. A table is normally
%not a spreadsheet! Compare \tref{tab:wrong} en \tref{tab:ok}: which table do
%you prefer?
%
%\begin{table}
%	\centering
%	\begin{tabular}{||l|lr||} \hline
%		gnats     & gram      & \$13.65 \\ \cline{2-3}
%		& each      & .01 \\ \hline
%		gnu       & stuffed   & 92.50 \\ \cline{1-1} \cline{3-3}
%		emu       &           & 33.33 \\ \hline
%		armadillo & frozen    & 8.99 \\ \hline
%	\end{tabular}
%	\caption{A table with the wrong layout.}
%	\label{tab:wrong}
%\end{table}
%
%\begin{table}
%	\centering
%	\begin{tabular}{@{}llr@{}} \toprule
%		\multicolumn{2}{c}{Item} \\ \cmidrule(r){1-2}
%		Animal    & Description & Price (\$)\\ \midrule
%		Gnat      & per gram    & 13.65 \\
%		& each        & 0.01 \\
%		Gnu       & stuffed     & 92.50 \\
%		Emu       & stuffed     & 33.33 \\
%		Armadillo & frozen      & 8.99 \\ \bottomrule
%	\end{tabular}
%	\caption{A table with the correct layout.}
%	\label{tab:ok}
%\end{table}


\section{Conclusion}
The final section of the chapter gives an overview of the important results
of this chapter. This implies that the introductory chapter and the
concluding chapter don't need a conclusion.

%%% Local Variables: 
%%% mode: latex
%%% TeX-master: "thesis"
%%% End: 

\chapter{Implementation}
\label{cha:5:impl}
\label{impl:Intro}
In this chapter we will discuss how we build our fuzzers, what issues we had to circumvent and how we did that. Starting off with how we got our seeds to fuzz upon, to then discuss how we implemented the three techniques to finally end with how we deobfuscated the found bugs.

\section{Software versions used}
\label{impl:softwareVersion}
Throughout this paper we used CPMpy\footnote{\url{https://github.com/CPMpy/cpmpy}} version V0.9.9 (commit \href{https://github.com/CPMpy/cpmpy/commit/e79b3afedc934a9437c2ddb3a9f54d7e2d7bd3ee}{e79b3af}), unless specified otherwise, this version was chosen as it was the latest release version at the time of testing the first technique. All techniques were developed in Python 3.8, the MiniZinc solvers came with MiniZinc Python\footnote{\url{https://github.com/MiniZinc/minizinc-python}} release version 0.7.0 (commit \href{https://github.com/MiniZinc/minizinc-python/commit/a195cf63fcfbc98665d70ab64efb5424db25bd7e}{a195cf6}). For the proprietary solver Gurobi\footnote{\url{https://www.gurobi.com/}}, we used its Python version 9.5.2 with an academic license. 
%problems Gurobi outputting Not_run this was a trial problem, with the academic version does not occure
Originally, we did try to utilize the trial version to ease possible reproducibility, but the restrictions on the complexity of the problems became a hindrance which resulted in us moving to an academic license. For the other versions of the solvers, we used the ones included in the already mentioned packages, except for MiniZinc’s transformations to Google’s OR-Tools\footnote{\url{https://github.com/google/or-tools}}, there we had to install OR-Tools for MiniZinc manually, which we did using release version 9.3.10497 (commit \href{https://github.com/google/or-tools/commit/49b6301e1e1e231d654d79b6032e79809868a70e}{49b6301}).


\section{Obtaining seeds}
\label{impl:obtainingSeeds}
As discussed in a previous section (Section \ref{fuzzing:generationMutation}) generating new inputs is significantly harder than mutation, but with the latter one we require a diverse set of seed files. Fortunately, the CPMpy team made a lot of documentation and examples on how to model problems in their language. Ranging from easy examples to teach the language to advanced examples in order to showcase certain features. 
At the moment of writing most examples are found in the main branch and some extras can be found on the “csplib” branch\footnote{\url{https://github.com/CPMpy/cpmpy/tree/csplib}} waiting to be merged with the main branch. We downloaded a copy of those branches on Tuesday 27th of September to be used as future seed files. 

A second source of seeds files came from Hakan Kjellerstrand a retired software developer and independent researcher from Sweden which was found while reading “Model-Based Algorithm Configuration with Adaptive Capping and Prior Distributions” \cite{18bleukx2022model}. Mr. Kjellerstrand has a big repository\footnote{\url{https://github.com/hakank/hakank/tree/master/cpmpy}} full of problem models which he solves in multiple ways, including CPMpy. We obtained a copy of all his CPMpy examples on Tuesday 27th of September to top off our collection of future seed files.


After that we ran all seeds to test that the non-modified seeds do not crash on their own and noticed that most examples ran in less than a minute. The handful of examples that did run long were left out or were simplified to gain a speed up while solving them. Knowing that all seeds are capable of being run in a minute helps us avoid the halting problem. If a modified seed starts running significantly longer than a minute we can start investigating that seed for a potential bug. A final change we made to the future seed files is extracting the model from each file found on the repositories. We did this for a couple of reasons: some files had a loop around the solve instruction combined with small changes or had multiple problems in one file, this gave us a separate model for each found problem. In order to extract these constraints, we temporarily modified CPMpy to extract the created model, variables and constraints included, each time solve was called, this resulted in over nine thousand problem models which we will use as our seed files.


%We extracted our seeds twice, a first time where we extract our model without any flattening of the constraints and a second time where we did flatten the constraints. While building up a model of the problem CPMpy allows for arbitrary complex compositions of constraints resulting in a nested tree of constraints. However, not all solvers allow this nested tree as described by the documentations of CPMpy. It is for that reason that CPMpy flattens the constraints to what they call ‘normal forms’ as the similar definition of SAT but with a disclaimer that this does not exits to their knowledge with which we agree with, with this flattened form CPMpy can directly call the solvers or do some changes for the solver interface on the flattened constraints to then send it to the solver. The reason we extracted our seeds with and without a flattening process is that \todo{aanvullen na vraag} a flattened version and the reason we did is without will become clear in the next section.

\section{Modifying STORM into CTORM}
\label{impl:modifyingSTROM}
Our first technique of finding bugs is heavily based on STORM which we shortly discussed before in Section \ref{fuzzing:testingWithFuzzers}. Instead of searching for SMT bugs, here we want to be able to find bugs in constraint programming languages and specifically in CPMpy. We downloaded STORM from the repository\footnote{\url{https://github.com/Practical-Formal-Methods/storm}} on Tuesday 27th September.
%https://github.com/Practical-Formal-Methods/storm/commit/55d091624523a0544112ffc339fe81103b3daa2b
The original plan was to convert our seeds to FlatZinc using the MiniZinc API provided by CPMpy to then convert that to SMT-LIB \cite{72bofill2010system} using Miquel Bofill et al.’s fzn2smt-tool to then be able to use STORM as it was built originally. Unfortunately (and a bit predictable), this way of working did not work out. On top of fzn2smt being more than a decade old, the multiple transformation layers that could introduce conversion bugs and the unclear way back from SMT-LIB prevented this path from being investigated by us.

Therefore, we decided to refactor STORM to fit CPMpy and name the technique CTORM for CPMpy-STORM. To change STORM into CTORM, we needed to rewrite the detection, labeling and construction of (sub)constraints, this refactoring did come with some downsides, some features of STORM no longer work such as incremental solving or the input obfuscation that was built-in. A bigger downside came with the refactoring of the negation function of STORM, as CPMpy is still in active development and 
the negation not always being implemented already, this was felt while trying to negate global constraints. I.e., when trying to invert (sub)constraints which include \texttt{alldifferent([var1, var2, var3])} using CPMpy, it crashed, this is of course a bug (more specifically not yet implemented) in CPMpy but used by the fuzzer. So here we had the choice of adding the missing negation of global constraints to CPMpy or to limit our fuzzer to not use the missing features. We choose to limit the fuzzer, since we are trying to detect bugs in CPMpy with different tools and extending the language ourselves goes out of scope of this thesis. 

%remove?
%The resulting limitation on our fuzzer only influences the speed of generating new constraints and it can theoretically now get stuck but this has not happened yet, so we believe it to be a acceptable limitation.


We gave only non-flattened inputs to this solver since both STORM and CTORM used a recursive process to get all subformulas because we wanted to change as little as possible to the inner workings of CTORM compared to STORM. The flattening process in CPMpy is used to convert the potential tree-like constraint structures to a flattened list of constraints which the solver can handle. 
In order to get those non-flattened inputs, we hijacked the flatten process of CPMpy to return all subformulas before returning the flattened constraints, this gave access to the more convoluted subformulas to use in the next steps of CTORM. This flattening process was done before any modifications were made, so in the eyes of the fuzzer it got flattened seeds but with the knowledge of some more complex constraints just like STORM and CTORM does. For each input CTORM combines 100 new constraints built from the existing constraints, this is repeated to create a hundred models to then check if the result matched with the original output in CPMpy.
%\todo{optional image of this CTORM processes?}

%DUMB ideas:
%Translating seeds from solver X to solver Y 
%option 1 hardcode default solver of CPMpy to Y, less good modifying the language is something we want to avoid. May also not work when solver is hard coded in the seed.
%Option 2 interpret the seed and make changes so that the solver Y is run. Bit trickier as you cannot see the difference between model.solve() and solver.solve() because model and solver are variables.
%
%nested functions with global function inside are giving problems


\section{Metamorphic testing}
\label{impl:Meta}
While CTORM was quite autonomous, metamorphic testing did take one step back to manual work, as this technique requires some metamorphic transformations, these transformations take a (or multiple) constraint(s) of our seed problems and change them repeatedly while keeping the (un)satisfiability the same. To then test if the original seed problem gives the same result as our modified problem, as discussed in Subsection \ref{fuzzing:MetamorphicTesting}. 

With papers such as \cite{50akgun2018metamorphic, 49usman2020testmc, 43YinYang} and others giving us inspiration, we came up with but not limited to the following 30 metamorphic relations. Replacing global constraints like \texttt{alldifferent([var1, var2, var3])} to their decomposition \texttt{var1!=var2 and var1!=var3 and var2!=var3}. Adding futile variables to global constraints such as \texttt{allequal([var4, var5])} by copying variables which did not limit (or restrict) the existing solution-space. We did this too for other more basic operations such as “and”, “or”, “xor”, “->” (implication), all forms of comparisons, min, max and others. We also included metamorphic relations proposed by the authors of “Validating SMT Solvers via Semantic Fusion” \cite{43YinYang}, those being semantic fusion for addition, subtraction, multiplication, and, or, xor and the comparisons. All analog to the example given in Subsection \ref{fuzzing:SemanticFusion}. 

We also linked multiple (sub)constraints of the problem to each other and replaced comparisons by other equivalent comparisons. Lastly, we also added new constraints which were independent of the original problem only to get in the way of the seed problem or be used in other metamorphic relations.

All these metamorphic relations individually were quite simple and should be handled easily by the flattening process, other CPMpy processes or by the solvers, but by combining multiple relations at random we were able to create more complex constraints that were not always handled correctly. Finally, we should note that while finishing this thesis Jo Devriendt found a bug within the metamorphic tester that made cyclic expressions possible which will crash CPMpy, which was swiftly fixed in the tester. More information about this bug can be found in the CPMpy bug report number \href{https://github.com/CPMpy/cpmpy/issues/163}{163}.

\section{Differential testing}
\label{impl:diff}
With differential testing we stepped a bit further away from the fuzzing world since we did not edit the input files.
With this last technique we put the (sub)solvers against each other, if solver A said that the problem is unsatisfiable and all other solvers said that the same problem was satisfiable we can say that we found a bug and that the bug was most likely to be in solver A, which differs from the previous techniques where we knew the correct solution in advance.

%This test could easily be integrated in one (or both) previous fuzzers 
% since this can be integrated as an extra check in the fuzzers. 

The way this tester was written is quite simple, we tested a given input on multiple solvers and searched if there was any difference in outputs or on the number of solutions provided in the results.
%ony 2 solvers
However, we discovered that only two solvers, namely “ortools” and “gurobi”-solvers, are able to search all solutions for problem models that contain global constraints. More solvers are available to find all solutions for SAT problems but 
the main objective of this thesis is to find CP-related bugs. The limitation to only two solvers results in only being able to compare two different implementations and has a risk of overlapping bugs. Preferably, we would have three or more solvers to be able to compare between them and also be able to automatically show us which of the solvers is likely to be wrong. In the future more solvers will be available within CPMpy with the capability of finding all solutions, but right now these tests are performed with two solvers. Small note, a limit of 100 solutions was put in place, otherwise finding solutions would take an unreasonable amount of time. 

When we look at only searching for one solution for a given example, we had more than enough solvers to compare between. Most of the times 14 solvers were compared and we never got lower than 6 solvers, this variation is due to the features used in the given problem the amount of solvers changed. If a problem happens to not use any global constraints the SAT-solvers may be able to solve them as well allowing us to compare between up to 36 solvers. Given that we only search for a single solution we are limited in what we can compare. For example, if solver A says that it found a satisfiable problem with the solution having values A' and solver B says the same but with different values B' (with A $\neq$ B ) the only thing we can compare is that both solvers output the same “satisfiable”. Since both solution A' and B' can be different solutions to the same problem.


\section{Detecting the cause of the bug}
\label{impl:DetectingCause}
As described in Chapter \ref{inputReduction:intro} finding a bug is one part of the problem, the other is finding which part of the inputs causes the bug. Submitting a complex input would result in a lot of work for the development team in order to find the cause. To avoid this situation, we used a Minimal Unsatisfiable Subset finder created by the CPMpy-team. The first version we used can be found in the advanced folder\footnote{\url{https://github.com/CPMpy/cpmpy/blob/master/examples/advanced/musx.py}} of the examples of CPMpy version 0.9.9. It was limited to finding MUS with the OR-Tools solver only, which caused us to write our own programs to deobfuscate the inputs based on this MUS-finder. 

We created a program to simplify inputs as described in Subsection \ref{inputReduction:Simplifying} and a program to isolate inputs as detailed in Subsection \ref{inputReduction:Isolation}. Nevertheless, we ended up only using the simplification technique due to not wanting to report a crucial part of the bug in one issue and reporting the difference between two inputs in the next issue. The already mentioned time penalty was not noticeable due to the inputs not being enormous. Both the programs we created did not give a minimal input back but did minimize the number of constraints, the difference being that some constraints could contain non-crucial parts of the bug. For example, a constraint “[15][variable1](...)” would give the exact same bug as constraint “[0, 0, 0, 0, 0, 0, 0, 0, 0, ... 0, 15][variable1](...)” but be clearer for the developer. 

To achieve a minimal input, we were planning to update our previously created programs, but CPMpy version 0.9.10 came out and contained a better MUS-finding program\footnote{\url{https://github.com/CPMpy/cpmpy/blob/master/cpmpy/tools/mus.py}} suitable for finding MUS for all available solvers, which was important since not all (sub)solvers would agree on the problem being satisfiable or not.


Nevertheless letting a solver search for an unsatisfiable subset in what it thinks is a satisfiable problem does not work, as it will never find an unsatisfiable subset for obvious reasons.
It was a similar case for crashes, for which we used a modified version of our simplification program. For obvious reasons the MUS-finders of CPMpy did not work for crashing programs. Our simplification program would test if the model still had a crash with some missing constraints or not and continue analog to the MUS version. It would again result in a minimal number of constraints but not in a minimal model. To get a minimal model some manual work was needed but that was not a significant amount of work. 

%\section{Tested solver}
%As a small side note not all solver within CPMpy are that usseefull if we wnat to find bug in CP languages 
%\todo{focused solvers}
%pySSD + PYsat: graph maker minder relevant omdat we CP fouten willen vinden 


\section{Conclusion}
\label{impl:conclusion}
In this chapter we discussed which software was used in tandem with the version’s numbers and GitHub repositories where applicable. We showed how we turned CPMpy’s and Hakan’s examples into the seed files we then used for our three techniques. After that, we specified how we implemented our techniques to find bugs, this being the modification of STORM into CTORM, the creation of metamorphic relations and the comparison of the output of (sub)solvers in the differential testing. To finally end with the tools used to minimize the input files once a bug was found.


%%% Local Variables: 
%%% mode: latex
%%% TeX-master: "thesis"
%%% End: 

% ... en zo verder tot
\chapter{The Final Chapter}
\label{cha:n}

\section{Conclusion}

%%% Local Variables: 
%%% mode: latex
%%% TeX-master: "thesis"
%%% End: 

\include{conclusion}

% Indien er bijlagen zijn:
\appendixpage*          % indien gewenst
\appendix
\chapter{Manual for setting up the fuzzer}
\label{app:A}


%%% Local Variables: 
%%% mode: latex
%%% TeX-master: "thesis"
%%% End: 

% ... en zo verder tot
\include{app-n}

\backmatter
% Na de bijlagen plaatst men nog de bibliografie.
% Je kan de  standaard "abbrv" bibliografiestijl vervangen door een andere.
%\bibliographystyle{abbrv}
%\bibliography{references}
\printbibliography
\end{document}

%%% Local Variables: 
%%% mode: latex
%%% TeX-master: t
%%% End: 

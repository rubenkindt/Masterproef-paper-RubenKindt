\chapter{Evaluation}
\label{cha:eval}
\label{cha:intro}
\todo{eval:intro}

% own results, multiple runs due to randomness, how measured
% comparision with other fuzzers

\section{Research questions}
\label{eval:RQ}
In this chapter we will try to formulate answers to the following research questions.
\subsection{fuzzing technique-focused}
RQ1: What fuzzing technique will find the most bugs?\\
RQ2: What fuzzing technique will find the most critical bugs?\\
RQ3: What type of bugs will be found using which fuzzing technique?\\
\subsubsection{subfocuses}
RQ4: Which metamorphic transformation find the most (critical) bugs?\\
focused on: \\
bool, int, list, array
\subsection{solver-focused}
RQ5: Which solver has the most (critical) bugs?\\
\subsection{classification-focused}
RQ6: How many (critical) bugs can we find?\\
RQ7: What the causes of the solvers\\
RQ8: What are the type of bugs found?\\
\subsection{efficient-focused}
RQ9: How efficient is the fuzzer?\\
RQ10: How efficient is the deobfuscator?\\
RQ11: How well does the deobfuscator deobfuscate the inputs?\\



\section{Conclusion}
\label{eval:conclusion}
\todo{conclusion}

%%% Local Variables: 
%%% mode: latex
%%% TeX-master: "thesis"
%%% End: 

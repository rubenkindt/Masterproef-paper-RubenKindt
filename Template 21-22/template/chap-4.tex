\chapter{CP and SMT}
\label{cha:4}
\todo{intro ch 4}

% see holy grail
\section{Holy grail of programming} %use as intro?
\begin{quote}
	"Constraint programming represents one of the closest approaches computer science has yet made to the Holy Grail of programming: the user states the problem, the computer solves it." \cite{11freuder1997pursuitHolyGrail} quote by Eugene C. Freuder
\end{quote}
As the quote and paper Eugene C. Freuder asserts, he believes that the 
% \cite{52bartak1999constraint}


%\cite{56bardin2019bringing} gives nice overview of constraint solving in general


\section{CP}
%\cite{53marriott1998programming}
% compaierde to traditional programming 1 relations expresseeed isnstead of 1 relation needing to be expreassed from all possible viewspoints (SEND + MORE = MONEY as example)

\subsection{origen}
%constraint satisfaction and constraint solving 
%Generate and Test is bad papaer 52
% CP is good at scheduling problems paper 52
%Limitations NP-hard and efficiency

\subsection{some Examples}
%minizinc also fits in SMT story \footnote{\url{https://www.minizinc.org/}} and \cite{57nethercote2007minizinc}
%std constraint programming frontend with multiple backend solvers
%check related work of 57 MiniZinc coms from Zinc ... (quote? or from conclusion)
%something about the challange \cite{58stuckey2014minizinc} "The principle aim of the MiniZinc Challenge is, un-surprisingly, to compare the state of the art among constraint-programming (CP) solvers. " paper 58
%others GeCode \footnote{\url{https://www.gecode.org/}}

\section{SMT}
%Z3 \cite{54moura2008z3} + \footnote{\url{https://github.com/Z3Prover/z3}}


\section{Conclusion}
\todo{Conclusion}
The final section of the chapter gives an overview of the important results
of this chapter. This implies that the introductory chapter and the
concluding chapter don't need a conclusion.

%%% Local Variables: 
%%% mode: latex
%%% TeX-master: "thesis"
%%% End: 

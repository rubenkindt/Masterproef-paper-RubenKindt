\chapter{CP, SAT and SMT}
\label{cha:CS}
\todo{intro ch CP and SMT}

% see holy grail
\section{Holy grail of programming} %use as intro?
\label{CS:HolyGrail}
\begin{quote}
	"Constraint programming represents one of the closest approaches computer science has yet made to the Holy Grail of programming: the user states the problem, the computer solves it." \cite{11freuder1997pursuitHolyGrail} quote by Eugene C. Freuder
\end{quote}
As the quote and the paper from Eugene C. Freuder says, he and others believe that in the ideal world the user conveys any problem to the program and that the program will solve it. Which does match a coarse summary of what a constraint programming language (CP) can do. But we don't live in an ideal world and problems need to be converted or split up into a representation the solver can understand. This is where constraints come into the picture, constraints are mathematical or logical limitations; or connections between variables when solved form a model that satisfy all constraints and sometimes minimize or maximize a final objective function, for example finding a model where a postman visits all cities on a route but minimize the distance traveled. 
With CP the focus lays more on the solving high level problems with specialties around scheduling and planning \cite{52bartak1999constraint}. Which their key feature being the global constraints, a set of "function" that are aimed at high level of abstraction. for example the "alldifferent()" which makes sure that all variables will receive a different value or "circuit()" that holds if the input forms a Hamiltonian circuit.

A second field of research that we will discuss is the boolean satisfiability problem better known as SAT, where the focus lays on the boolean variant of constraints within the family of constraint solvers. This field of research has produced quite a lot of progress due to its age, resulting in efficiency in solving \cite{56bardin2019bringing}.

Our last field of research that we will discuss, is the satisfiability modulo theory (SMT) which builds further on SAT. This by including Lists, arrays, strings, real numbers, integers and other more complex data structures and types. With SMT the focus lays more on static checking and program verification \cite{56bardin2019bringing, 54moura2008z3}.


\section{Constraint programming}
\label{CS:CP}
As mentioned before CP's are well versatile in the solving of constraints and especially when it comes to planning and scheduling. This by their efficient constraint propagation, backtracking and the linking of related constraints \cite{66WikiCP}. Originally CP's can be linked back to constraint logic programming (CLP) where programming languages (mostly logic programming languages) where combined with constraints. A notable version is that of Joxan Jaffart and Jean-Louis Lassez \cite{65jaffar1987constraint, 66WikiCP} extension in Prolog.

To further introduces you to constraint programming we will show a popular puzzle in the CP field "send + more = money". which is a logic puzzle made by Henry Dudeney and published in Strand Magazine's July 1924 edition \cite{sendMoreMoney}.
Where each character represents a single digit between zero and nine (both included), meaning that the 'e' from "send" should be the same value as the 'e' from "more" and "money". The other rules go as follows: all characters should have a different digit, the starting letter of each word can not be zero and after replacing all characters with digits the sum should be correct. 
\begin{center}
	\[$$
	\; \; S E N D\\
	+ M O R E\\
	-----------------\\
	M O N E Y
	$$\]
\end{center}
This example is often used to introduce CP field, so will we. We will also use MiniZinc\footnote{\url{https://www.minizinc.org/}} as our constraint programming language to find the solution for this example.

\label{lst:SendMoreMoney}
\begin{lstlisting}[language=minizinc, caption={Solution to "send + more = money" slightly modified from \url{https://www.minizinc.org/doc-2.5.5/en/downloads/send-more-money.mzn}}]
	include "globals.mzn";
	var 1..9: S; % Since 'S' is a starting chacter it is limiter from 1 to 9 included.
	var 0..9: E; % Other charcaters are limited from 0 to 9 included.
	var 0..9: N;
	var 0..9: D;
	var 1..9: M;
	var 0..9: O;
	var 0..9: R;
	var 0..9: Y;
	
	constraint % The sum must hold.
	              1000 * S + 100 * E + 10 * N + D
	            + 1000 * M + 100 * O + 10 * R + E
	= 10000 * M + 1000 * O + 100 * N + 10 * E + Y;
	
	constraint alldifferent([S,E,N,D,M,O,R,Y]);
	
	solve satisfy;
	output 
	[" \(S)\(E)\(N)\(D)\n",
	"+ \(M)\(O)\(R)\(E)\n",
	"= \(M)\(O)\(N)\(E)\(Y)\n"];
\end{lstlisting}
\label{sendMoreMoneyExplanation}
On line 1 you can see the importing of the global constraints, which CP is know for. Line 2 until 9 we can see the declarations of all possible characters to the possible digits, starting letters like 'S' and 'M' are limited by their representation from 1 to 9 instead of using a constraint. The constraint on line 11 runs through until line 14, where it is closed by a the semicolon. This constraint specifies the matching of the sum. At line 16 we use one of the imported functions namely "alldifferent()" which will satisfy if no similar values occur in the array. To then start solving for satisfiability at line 18, this in comparison of solving for minimization or maximization of an objectify function, which does not apply here. Finally after a solver has found satisfiability we print the result using a pretty print from line 19 to 22. We will leave it up to you for finding the solution of the puzzle, but remember that you can check your answer with the program above.

The advantage of using constraints is also visible by the sum constraint, if we had used allocations 
The "alldifferent" and the sum constraint also shows us the power of CP's, with a single statement multiple relations are expressed\cite{53marriott1998programming}. Instead of needing to specify all possible relations of the last tree characters (D + E = Y, E = Y - D and D = Y - E) a single expression suffices. When knowing two values we can infer the third, imagine the work needed for the "alldifferent()" constraint. 

\subsection{origin}
Within constraint programming we can distinguish two branches \cite{52bartak1999constraint}: one being constraint satisfaction. Which puts the focus on finding a model(s) which satisfies all constraint. which can be done with generating values within the domain of the variables and testing them, also called generate and test, but obviously this is not the fastest way. 

On the other hand we have constraint optimization, which covers an even harder problem. Instead of having to check if a constraints satisfies, here we want to know what model will give us the best value. The optimization function is often called the objective function and occurs quite often in real life problems \cite{52bartak1999constraint}. Unfortunately depending on the problem we often hit limitations due to NP-hardness. Different search strategies can help to gain a higher efficiency and a lot have been though off. Popular approaches to finding solutions in both branches are the use of: constraint propagation, backtracking, symmetry breaking, dynamic programming, techniques from the SAT solvers like lazy clause generation (more on that later) and even heuristics as: local search, Tabu search, simulated annealing and more. 

\subsection{MiniZinc}
A keen observer has will have noticed that we used "after \underline{Underlined text} a solver has found satisfiability" in the explanation of "send + more = money" \ref{sendMoreMoneyExplanation}. This is because MiniZinc is not a solver, it came to be from the lack of standard modeling language surrounding CP's. When you wanted to use another solver you had to model you problem again in that solver's specific language. This is what Nicholas Nethercote et al. wanted to solve, they came up with MiniZinc which is a modeling language for CP's that is not connected to a single specific solver \cite{57nethercote2007minizinc}. 

In the words of Peter J. Stucke, a member of the MiniZinc team: 
\begin{quote}
	"MiniZinc is high level enough to express most combinatorial optimization problems easily and in a largely solver-
	independent way; (...) However, MiniZinc is low level enough that it can be mapped easily onto many solvers." \cite{58stuckey2014minizinc}
\end{quote}
MiniZinc transforms its inputs to FlatZinc which then can be solved by a growing list of CP solvers. At the time of writing 17 solvers have FlatZinc interfaces. MiniZinc originated from a modeling languages focused on constraints, called Zinc \cite{68incbanda2006modelling}. As you can tell by its name MiniZinc is a subset of Zinc \cite{57nethercote2007minizinc}.

\subsubsection{MiniZinc Challenge}
An intentional benefit of having a standardized constraint programming language is the ability to benchmark CP solvers. Which the MiniZinc team did with the organization of the "MiniZinc Challenge". In this yearly challenge a solver gets  15 minutes to solve a 100 selected problem instances to solve. Due the difficulty of having to find quite an amount of good representative problem instances, the organizers of the challenge ask the participants to submit two problem models and multiple related instances. From the received list the jury then tries to make a fair selection to cover the use of: global constraints, real-world representative problems end to find a good balance between satisfying versus optimization problems and the different types of technologies (SAT and MIP) instead of only selecting CP focused problems \cite{58stuckey2014minizinc}.

It is due to this MiniZinc challenge that a better connection has formed between CP solvers and SAT solvers. By attracting SAT solvers to the challenge a better connection forms between the fields of research, like BEE \cite{69BEEmetodi2012compiling} which is a compiler able to translate FlatZinc to conjunctive normal form (CNF). From this CNF Amit Metodi and Michael Codish where then able to let a SAT solver solve the problem. Although it is limited to finite domain constraint problems this allows us bring the field of research closer.

On top of being a great way to benchmark comparative solvers the MiniZinc challenge, it also results in more solver implementing the FlatZinc as an input and due to the competitive nature of academics bringing the motivation to stride forwards according to the author of "Philosophy of the MiniZinc challenge", Julien Fischer \cite{59stuckey2010philosophy}.

\subsection{CPMpy}
\label{CP:CPMpy}
MiniZinc is not de only one with the idea to create a modeling language for CP's, other examples are Essence with the focus on combinatorial problems for people with a background in discrete mathematics \cite{70frisch2008essence}.
And the one with our most interest is the constraint programming and modeling language for Python (CPMpy) based around the popular packet NumPy. By creating a constraint programming and modeling language in Python the learning curve of CP's becomes lower. On top of that by facilitating the use of NumPy, which is popular in the data processing and the general array based operations, CPMpy makes a familiar data processing tool available to be used while programming constraints \cite{17guns2019increasing}.

\todo{option to multiple solvers}

We would have wanted to write an in-depth explanation on how CPMpy doe its conversions from constraints to what is given to the solvers, but CPMpy being in beta and with more changes potential coming this could become outdated quickly. We do look forward to the team working on CPMpy and their future paper(s) explaining it.

\section{SAT}
\label{CS:SAT}
\todo{lazy clause generation example?} %\cite{67stuckey2010lazyClauseGeneration}

%\cite{60katebi2011empirical}

\section{SMT}
\label{CS:SMT}
%Z3 \cite{54moura2008z3} + \footnote{\url{https://github.com/Z3Prover/z3}}
%CVC4 CVC5 \footnote{\url{https://github.com/cvc5/cvc5}} the fifth in line of the build Cooperating Validity Checkers (CVC), the previous where (CVC, CVC Lite, CVC3 end CVC4) \cite{62barrettcvc5} \cite{63barbosa2022cvc5}
%mention techniques branching, unit propagation, backtracking, DPLL, CDCL, 2 watched literals, heuristics: detpth, breadth first, Variable state independent decaying sum/VSIDS, dynamic largest individual sum (DLIS), clause deletion, random restarts \cite{60katebi2011empirical} \cite{61MCSMarcDenecker}
% CVC5 has API's for C++, Python and Java
%Z3 API's conatins but not limited to C/C++, .NET, OCaml, Python, Java, Haskell \cite{64WikiSMT}


\section{Conclusion}
\label{CS:conclusion}
\todo{Conclusion}
The final section of the chapter gives an overview of the important results
of this chapter. This implies that the introductory chapter and the
concluding chapter don't need a conclusion.

%%% Local Variables: 
%%% mode: latex
%%% TeX-master: "thesis"
%%% End: 

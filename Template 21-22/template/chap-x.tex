\chapter{creation of fuzzer}
\label{cha:x}
\todo{intro ch x}

% own results, multiple runs due toe randomness, how measured
% comparision with other fuzzers

\section{creation of sat and unsat formulas}
see paper 43 p 4
Semantic Fusion
\cite{43YinYang}
een ref to chapter 2

seed files came from the CPMpy repository the main branch and the csplib branch aswell from Hakank's repository downloaded on Tuesday 27/09/2022
Storm downloaded on Tuesday 27/09/2022 from https://github.com/Practical-Formal-Methods/storm 
https://github.com/Practical-Formal-Methods/storm/commit/55d091624523a0544112ffc339fe81103b3daa2b
We also made sure that the seed files were able to be run in less then 60 sec either by reducing the amount of solver calls per example of in extreme situations removed some examples


Storm take in SMT-lib seed files, we can convert them to minizinc and then using fzn2smt but It hasn't been maintained in over a decenta and doing multiple convertions only back could be tricky and would introduce multiple layers which can introduce bugs a normal user would ever see
Therefore we will be refactoring STORM to fit our CPMpy


Translating seeds from solver X to solver Y: 
option 1 hardcode default solver of CPMpy to Y, less good modifying the language is something we want to avoid. May also not work when solver is hard coded in the seed.
Option 2 interpret the seed and make changes so that the solver Y is run. Bit trickier as you cannot see the difference between model.solve() and solver.solve() because model and solver are variables.






















%
%\section{Figures}
%Figures are used to add illustrations to the text. The \fref{fig:logo} shows
%the KU~Leuven logo as an illustration.
%\begin{figure}
%	\centering
%	\includegraphics{logokul}
%	\caption{The KU~Leuven logo.}
%	\label{fig:logo}
%\end{figure}
%
%\section{Tables}
%Tables are used to present data neatly arranged. A table is normally
%not a spreadsheet! Compare \tref{tab:wrong} en \tref{tab:ok}: which table do
%you prefer?
%
%\begin{table}
%	\centering
%	\begin{tabular}{||l|lr||} \hline
%		gnats     & gram      & \$13.65 \\ \cline{2-3}
%		& each      & .01 \\ \hline
%		gnu       & stuffed   & 92.50 \\ \cline{1-1} \cline{3-3}
%		emu       &           & 33.33 \\ \hline
%		armadillo & frozen    & 8.99 \\ \hline
%	\end{tabular}
%	\caption{A table with the wrong layout.}
%	\label{tab:wrong}
%\end{table}
%
%\begin{table}
%	\centering
%	\begin{tabular}{@{}llr@{}} \toprule
%		\multicolumn{2}{c}{Item} \\ \cmidrule(r){1-2}
%		Animal    & Description & Price (\$)\\ \midrule
%		Gnat      & per gram    & 13.65 \\
%		& each        & 0.01 \\
%		Gnu       & stuffed     & 92.50 \\
%		Emu       & stuffed     & 33.33 \\
%		Armadillo & frozen      & 8.99 \\ \bottomrule
%	\end{tabular}
%	\caption{A table with the correct layout.}
%	\label{tab:ok}
%\end{table}
%

\section{Conclusion}
The final section of the chapter gives an overview of the important results
of this chapter. This implies that the introductory chapter and the
concluding chapter don't need a conclusion.

%%% Local Variables: 
%%% mode: latex
%%% TeX-master: "thesis"
%%% End: 

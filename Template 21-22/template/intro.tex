\chapter{Introduction}
\label{cha:intro}
Programmers make mistakes, just like everyone.
Software is often complex, written by multiple people, used with the wrong assumptions or does not meet the objective goals. 
There are a lot of causes for bugs: software complexity, multiple people writing different parts of software, changing objective goals of software, misaligned assumptions and more. Most these things can not be avoided during the creation of software but do cause program crashes, vulnerabilities, wrong outcomes and more.
Multiple forms of prevention have been created in various forms of software testers, documentation, automatic tests, code reviews. All of these aim to prevent the occurrence of bugs. While automatic test cases often evaluate goals of software end evaluate previous known bugs, it can do much more.
Fuzzing software is one of those things, a technique that is popular in the security world for exploit prevention, which generates random input for a program under test (PUT) and monitors if the program crashes or not. This explanation was the original interpretation of fuzzing as preformed by \cite{originalFuzzingUnixUtils}, today this technique is seen as random fuzzing and fuzzing envelops a broader term, as V. J. Man\`es et al. the authors of \cite{13manes2019survey} put it nicely:
\begin{quote}
"Fuzzing refers to a process of repeatedly running a program with generated inputs that may be syntactically or semantically malformed."
\end{quote} as quoted from \cite{13manes2019survey}.

With this technique we will try to detect bugs in the constraint programming and modeling library CPMpy \cite{guns2019increasing} created by Prof. dr. Guns et al..


The first contains a general introduction to the work. The goals are\todo{remove this} defined and the modus operandi is explained.

%%% Local Variables: 
%%% mode: latex
%%% TeX-master: "thesis"
%%% End: 

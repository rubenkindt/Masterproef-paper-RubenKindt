\chapter{Introduction}
\label{cha:intro}
Programmers make mistakes, just like everyone.
Software is often complex, written by multiple people, sometimes used with the wrong assumptions or does not meet the objective goals. 
There are a lot of causes for bugs: software complexity, multiple people writing different parts of software, changing objective goals of software, misaligned assumptions and more. Most these things can not be avoided during the creation of software but do cause program crashes, vulnerabilities, wrong outcomes and more.
Multiple forms of prevention have been created in various forms of software testers, documentation, automatic tests, code reviews. All of these aim to prevent the occurrence of bugs. While automatic test cases often evaluate goals of software, previous known bugs, they can do much more.



The first contains a general introduction to the work. The goals are\todo{remove this} defined and the modus operandi is explained.

%%% Local Variables: 
%%% mode: latex
%%% TeX-master: "thesis"
%%% End: 

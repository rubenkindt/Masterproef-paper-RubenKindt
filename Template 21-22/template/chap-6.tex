\chapter{Results}
\label{cha:6}
\todo{intro ch 6}

%serverspecs
%time when faults are found?

\section{STORM}
Techniques are best applied during development, since the modified storm of to often detects the already fond bugs

%problems limited by global fucntions of used seeds (somewhat fine),
%only 'and' and 'not' combinations Just like STORM

~(~(True))
~(global function)
\section{Differential testing}
solverlookup()  was during development = manual testing

\section{Metamorphic testing}
did not find ~(~()) simly because we dind't think to check it specificly, we did check ~=0
all checks ? + combo-able

\section{YinYang}


\section{unsat}
due to the way of importing the file a lot of edge problems become sat or unsat

\section{woringly sat}
meta unsat minizinc 0
to shirk this we used the fact that ortools+gurobi (probably) gave the correct solution, this bieing unsat and taking the mus from that


\subsection{finding mus}
combination of theire tool, but when we got errors from the different way of loading, We used my tool en then theirs. general combinations were done got get a minimal


\section{title}
%these tools often gave the same error again causing the new bug to be hidden, se choice to catch some of the more prevelent bugs, If some one wishes to rerun (parts) It is best to remove those exceptions in order to see the full output


\section{Conclusion}
\todo{Conclusion}
The final section of the chapter gives an overview of the important results
of this chapter. This implies that the introductory chapter and the
concluding chapter don't need a conclusion.

%%% Local Variables: 
%%% mode: latex
%%% TeX-master: "thesis"
%%% End: 

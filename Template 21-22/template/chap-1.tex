\chapter{Fuzzing}
\label{cha:1}
A chapter is a logical unit. It normally starts with an introduction, which
you are reading now. The last topic of the chapter holds the conclusion.
\todo{intro chapter 1}

%if inf seed set, reducing it is better with the following technique (the min seed subset without any code coverage loss) \cite{14rebert2014seedselecting}
%seed selection problem, inf seeds which one do we use
%other fuzzers internal workings
%Chapter around fuzzing
%	\cite{mathis2019parser} states 3 optins traditional, stochastic and syntax driven.
%avoiding inf loops -> timeouts
%
%subsection about (psuedo-)randomness

\section{Seed selection}
%Please don't abuse enumerations: short enumerations shouldn't use
%``\verb|itemize|'' or ``\texttt{enumerate}'' environments.
%So \emph{never write}: 
%\begin{quote}
%  The Eiffel tower has three floors:
 % \begin{itemize}
 % \item the first one;
 % \item the second one;
 % \item the third one.
  %\end{itemize}
%\end{quote}
%But write:
%\begin{quote}
%  The Eiffel tower has three floors: the first one, the second one, and the
%  third one.
%\end{quote}

\section{A Second Topic}


\subsection{Another item}


\section{Conclusion}
The final section of the chapter gives an overview of the important results
of this chapter. This implies that the introductory chapter and the
concluding chapter don't need a conclusion.


%%% Local Variables: 
%%% mode: latex
%%% TeX-master: "thesis"
%%% End: 

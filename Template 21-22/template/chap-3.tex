\chapter{Simplifying crashes}
\label{cha:3}

%chapter on simplifying the crashes
%	binary search will not work all the time
%	quarters remove may work (if all parts fail go more granular, 1/9 or 1/16)
%		start with halfs then *2
%		always search further with same granularity but with removed part until all options with that granularity searched \cite{zeller2009programs} p111
%		this uses no knowledge from input structure and program structure \cite{zeller2009programs} p112
%	delta debugging
%		time spend searching vs simplified ratio is important as mentioned in \cite{mansur2020detecting}
%		and needs to preserve satisfiability as mentioned in \cite{mansur2020detecting}
%		^ possibly a big deal to find critical bugs
%
%	with knowledge of input, syntax \cite{zeller2009programs}
%	of by bigger entities like lines of words \cite{zeller2009programs}
%	 for speed
%
%	alt approach like \cite{mansur2020detecting} try finding the bug again with less resources avail
%	or isolaytion \cite{zeller2009programs} p 285 
%		I think it may fail if multiple parts are relevant
%		I think it could detect for example the CPMpy import as a bug cause as the min diff that causes the bug
%
%	sub section on MUS/minimum unsat subset vs delta debugging
%		MUs good for only whole constraints while 
%		delta debugging goes for partial structures
%
%	%dedublication techniques: stack backtrace (hashing), coverage beased, semantics aware \cite{13manes2019survey}
%	isolation vs simplifying \cite{zeller2009programs} p 285
%
%The Precision Effect minimizing/simplyfying may lead to a dif found bug if multiple, since all need to be solved not problem  \cite{zeller2002simplifying}
%	 can be solved with stack trace comparison
%
%section on minimalisation vs isolation \cite{zeller2002simplifying} p12
%	for speed take smaller inputs
%	Isolation will require a lot more things to track but is faster



\section{The First Topic of this Chapter}

\subsection{An item}
A master's thesis is never an isolated work. This means that your text must
contain references. On-line documents\cite{wiki} as well as
books\cite{pratchett06:_good_omens} can be referenced.

\section{Figures}
Figures are used to add illustrations to the text. The \fref{fig:logo} shows
the KU~Leuven logo as an illustration.
\begin{figure}
  \centering
  \includegraphics{logokul}
  \caption{The KU~Leuven logo.}
  \label{fig:logo}
\end{figure}

\section{Tables}
Tables are used to present data neatly arranged. A table is normally
not a spreadsheet! Compare \tref{tab:wrong} en \tref{tab:ok}: which table do
you prefer?

\begin{table}
  \centering
  \begin{tabular}{||l|lr||} \hline
    gnats     & gram      & \$13.65 \\ \cline{2-3}
              & each      & .01 \\ \hline
    gnu       & stuffed   & 92.50 \\ \cline{1-1} \cline{3-3}
    emu       &           & 33.33 \\ \hline
    armadillo & frozen    & 8.99 \\ \hline
  \end{tabular}
  \caption{A table with the wrong layout.}
  \label{tab:wrong}
\end{table}

\begin{table}
  \centering
  \begin{tabular}{@{}llr@{}} \toprule
    \multicolumn{2}{c}{Item} \\ \cmidrule(r){1-2}
    Animal    & Description & Price (\$)\\ \midrule
    Gnat      & per gram    & 13.65 \\
              & each        & 0.01 \\
    Gnu       & stuffed     & 92.50 \\
    Emu       & stuffed     & 33.33 \\
    Armadillo & frozen      & 8.99 \\ \bottomrule
  \end{tabular}
  \caption{A table with the correct layout.}
  \label{tab:ok}
\end{table}

\section{Lorem Ipsum}
This section is added to check headers and footers. So this chapter must at
least contain three pages. To make sure that we get the required amount,
the \textsf{lipsum} package isn't used but the text is put directly in the
text.

\subsection{Lorem ipsum dolor sit amet, consectetur adipiscing elit}

\subsection{Praesent auctor venenatis posuere}

\subsection{Morbi et mauris tempus purus ornare vehicula}


\section{Conclusion}
The final section of the chapter gives an overview of the important results
of this chapter. This implies that the introductory chapter and the
concluding chapter don't need a conclusion.


%%% Local Variables: 
%%% mode: latex
%%% TeX-master: "thesis"
%%% End: 

\chapter{Simplifying Inputs}
\label{cha:3}
When we detect that the PUT crashes, wrongly satisfies, wrongly not satisfies or hangs on a given input we now want to know why it does that. What causes this unwanted output and what line the bug occurs. With crashes, a stack trace and some luck this could be easy, but when the crash is not main perpetrator or we get an other unwanted output the developer my need to debug deep into the code to find the bug. This with a potential large input could be a tedious assignment, for this reason we would like to know what parts of the input are related to the bug. We will discover this further in this chapter, starting with \todo{plug in sections}

\section{Minimalisation and isolation}
% what are both, what situations is which the better one
%section on minimalisation vs isolation \cite{zeller2002simplifying} p12
%	for speed take smaller inputs
%	Isolation will require a lot more things to track but is faster


%small by construction source 9


%chapter on simplifying the crashes
%	binary search will not work all the time
%	quarters remove may work (if all parts fail go more granular, 1/9 or 1/16)
%		start with halfs then *2
%		always search further with same granularity but with removed part until all options with that granularity searched \cite{zeller2009programs} p111
%		this uses no knowledge from input structure and program structure \cite{zeller2009programs} p112
%	delta debugging
%		time spend searching vs simplified ratio is important as mentioned in \cite{mansur2020detecting}
%		and needs to preserve satisfiability as mentioned in \cite{mansur2020detecting}
%		^ possibly a big deal to find critical bugs
%
%	with knowledge of input, syntax \cite{zeller2009programs}
%	of by bigger entities like lines of words \cite{zeller2009programs}
%	 for speed
%
%	alt approach like \cite{mansur2020detecting} try finding the bug again with less resources avail
%	or isolaytion \cite{zeller2009programs} p 285 
%		I think it may fail if multiple parts are relevant
%		I think it could detect for example the CPMpy import as a bug cause as the min diff that causes the bug
%
%	sub section on MUS/minimum unsat subset vs delta debugging
%		MUs good for only whole constraints while 
%		delta debugging goes for partial structures
%

\section{Deduplication}
Another thing to notice is that multiple inputs could prompt the same bug from occurring, these inputs could be similar but don't have to be. With simplifying the input we should be able to detect exact copies, but depending on the simplification's time complexity other techniques could be better with similar results. In case where we would have access to stack traces (via crashes or hanging PUT's) we could differentiate the bugs on basis of the hash of multiple lines of the backtrace. this technique is called stack backtrace hashing and is quite popular according to Valentin J.M. Man\`es\cite{13manes2019survey}. Another technique talked about in that paper, is looking at the code coverage generated by the inputs where we use the executed path (or hash of it) is used as a fingerprint of the inputs. A last technique, used by Microsoft\cite{36semanticsAwareDeduplicationRETracer} is called semantics based deduplication, where in stead of back track use memory dumps to hopefully find the origins of bugs. This use of dumps is less ideal due to traces having more information, but the latter is not always possible due to the performance overhead and privacy causes as specified in the paper. None of the techniques mentioned above are perfect: with stack backtrace hashing you could find to many false positives or false negatives depending on the depth taken from the stack, with coverage some inputs will generate extra function calls and the semantics based deduplication are limited to X86 or x86-64 code with the binary file and the debug information. Neither of these techniques work with black box fuzzing unfortunately.
\todo{lookup other deduplication techniques}

%	%dedublication techniques: stack backtrace (hashing), coverage beased, semantics aware \cite{13manes2019survey}
%	isolation vs simplifying \cite{zeller2009programs} p 285
%
%The Precision Effect minimizing/simplyfying may lead to a dif found bug if multiple, since all need to be solved not problem  \cite{zeller2002simplifying}
%	 can be solved with stack trace comparison
%



\section{The First Topic of this Chapter}
%
%\section{Figures}
%Figures are used to add illustrations to the text. The \fref{fig:logo} shows
%the KU~Leuven logo as an illustration.
%\begin{figure}
%  \centering
%  \includegraphics{logokul}
%  \caption{The KU~Leuven logo.}
%  \label{fig:logo}
%\end{figure}

%\begin{table}
%  \centering
%  \begin{tabular}{@{}llr@{}} \toprule
%    \multicolumn{2}{c}{Item} \\ \cmidrule(r){1-2}
%    Animal    & Description & Price (\$)\\ \midrule
%    Gnat      & per gram    & 13.65 \\
%              & each        & 0.01 \\
%    Gnu       & stuffed     & 92.50 \\
%    Emu       & stuffed     & 33.33 \\
%    Armadillo & frozen      & 8.99 \\ \bottomrule
%  \end{tabular}
%  \caption{A table with the correct layout.}
%  \label{tab:ok}
%\end{table}

\section{Conclusion}
The final section of the chapter gives an overview of the important results
of this chapter. This implies that the introductory chapter and the
concluding chapter don't need a conclusion.


%%% Local Variables: 
%%% mode: latex
%%% TeX-master: "thesis"
%%% End: 

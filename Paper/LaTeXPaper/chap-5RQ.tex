\chapter{Research questions}
\label{cha:RQ}
\label{RQ:intro}
In this chapter we will discuss the problem we are facing today and where our focus will be when answering the research questions in this dissertation. \todo{Add this chapter to Goals of introductuion}

\section{Problem statement}
\label{RQ:ProblemStatment}
As described in at the introduction of chapter \ref{intro:intro}, bugs are practically unavoidable and always unwanted, especially when a user trusts a program to give a correct answer and it does not. With solvers surrounding constraint programming languages being executed more and more we would like to strongly avoid any bugs in the real world from arising. To this end it would be interesting to find bugs during development without much overhead, a modern approach would be the use of fuzzers. which we will try out on a constraint programming language.

\section{Research questions}
\label{RQ:RQ's}
As the title of the dissertation already may have spoiled it, we are trying out multiple fuzzing techniques out on CPMpy, with the goal of finding which technique works well for this specific type of programming language. This in order to give a push to identify ways of automatically discovering (and maybe solving) new bugs in constrain programming languages. We put forward two regions of research questions we will to focus on.

%\todo{reduce nb of RQ}
\subsection{Main focus: fuzzing technique-focused}
The first and our main focus will be comparing different fuzzing techniques: we are going to modify a successful SMT fuzzer STORM to the CPMpy language, try differential testing between the multiple solvers and out last technique is the use of metamorphic testing. Resulting in the following questions: \newline
Research question 1: What fuzzing technique will find the most bugs? \newline 
Research question 2: What fuzzing technique will find the most critical bugs? \newline
Research question 3: What type of bugs will be found by each fuzzing technique? \newline
%\subsubsection{subfocuses}
%Research question 4: Which metamorphic transformation find the most (critical) bugs? \newline
%focused on: \\
%bool, int, list, array

%\subsection{Solver-focused}
%A second focus we have goes more towards the different solvers and the differences between them resulting in. \newline
%Research question 5: Which solver has the most (critical) bugs? \newline

\subsection{Classification-focused}
Our next and last focus will be on the classification of found bugs, giving us the following research questions. \newline
%To then end with focus three based around the classification of found bugs, giving us the following research questions. \newline
Research question 4: How many (critical) bugs can we find? \newline
Research question 5: What are the causes of the bugs? \newline
%Research question 8: What are the type of bugs found? \newline

\section{Not focused: efficiency and others}
A keen reader may wonder why we do not focus on efficiency, this would result in more bugs being caught in a smaller timeframe. While perfectly valid to investigate we believe that discovering which techniques works best for CP' has a higher value than on top of the techniques investigated being able to run automatically. The efficiency in CP has already significant research and literature on how to optimize the solvers, which take the most amount execution time compared to the testers.

% all written in Python

\section{Conclusion}
\label{RQ:conclusion}
In this short chapter we have seen the problem which to address and the research questions this dissertation will answer in order to do so.

%%% Local Variables: 
%%% mode: latex
%%% TeX-master: "thesis"
%%% End: 

\chapter{Introduction}
\label{cha:intro}
\label{intro:intro}
There are a lot of causes for bugs software complexity, multiple people writing different parts, changing objective goals, misaligned assumptions and more. Most these things cannot be avoided during the creation of software but are the cause of program crashes, vulnerabilities or wrong outcomes. Multiple forms of prevention have been created like the various forms of software testing, documentation, automatic tests and code reviews. All with the aim to prevent the occurrence of bugs and to reduce the cost associated with them. While automatic test cases often evaluate the goals of software end evaluate previous known bugs, it can do much more. Fuzzing software is a part of those automatic tests, a technique that is popular in the security world for exploit prevention. This technique generates random input for a program under test (PUT) and monitors if the program crashes or not. This explanation was the original interpretation of fuzzing as performed by Miller \cite{4originalFuzzingUnixUtils}, today this technique is seen as random generation based black box fuzzing while the current fuzzing envelops a broader term, as Man\`es et al. \cite{13manes2019survey} put it nicely,
\begin{quote}
"Fuzzing refers to a process of repeatedly running a program with generated inputs that may be syntactically or semantically malformed."
\end{quote}, as quoted from \cite{13manes2019survey}.
With this technique we will try to detect bugs in the constraint programming and modeling library CPMpy \cite{17guns2019increasing} created by professor dr. Guns et al.

\section{The usage of fuzzers in the software development cycle}
\label{intro:SoftwareDevelopmentCycle}
During the development phase of software, tests are performed to check if the written code matches the expected and wanted output. This can be done by the developers themselves or by quality assurance testers which do this full time and this on multiple different ways, for example by code review, manual testing or automated testing. All these techniques could exist out of unit tests, checking for known bugs, regression testing, confirming that the use cases are working, code audits, dynamic testing, fuzzing and others. None of the techniques mentioned can prevent all possible bugs from occurring and using only a single technique would cost more to find the same level of bugs then using a combination of multiple techniques. Sometimes a code audit is better, for example in situations where you want to know something easy that is most likely plainly written in the code. Other cases dynamic testing may be better, image having a program which parses curricula vitae to check if candidates match the job position and you want to check if a fresh computer science graduate fit the position of software analyst. In this case it may be a lot easier to test some curricula vitae than to dive into the code. In situations where you want to test if bugs exist, you may not know where to start inside of the program under test (PUT), this is where fuzzing may be the correct tool to use. By submitting random inputs into the PUT and looking at the next actions the program takes (i.e., does it crash, give wrong results or other unwanted actions) the fuzzer can automatically detect bugs.

Fuzzing emerged in the academic literature at the start of the nineties, while the industry's full adoption thirty years later is still ongoing. Multiple companies like Google, Microsoft and LLVM have created their own fuzzers and this together with a pushing security sector for the adoption has caused fuzzing to become a part of the growing toolchain for software verification.

\section{Fuzzing and security}
\label{intro:FussingSecurity}
Fuzzing is a novel way for attempting to automate the finding of bugs and eventually some of these bugs will be security related. Depending on the application of the program and its environment it is either problem or not. But those security related bugs could be costly as discussed in the previous section, bugs become more costly the later you 
catch them. With security bugs being seen as the pricier ones, i.e., you do not want your company to get hacked, sued or being featured in the news for being exploited on top of the normal cost of having to find and fix the bug. %\todo{could not find source that says security bugs are more costly}
Ideally a company should not have to spend time, energy and money into finding and fixing bugs, but there will be miscommunications, mistakes and more that will result in bugs. Therefore, companies will need to invest in prevention and (early) detection. Which is shown in the adoption of fuzzers that has gained speed due to its proven effectiveness in finding security related bugs. For example, ShellShock, Heartbleed, Log4Shell, Foreshadow and KRACK could have been found using fuzz testing as shown in multiple sources \cite{HeartbleedViaFuzzing, 34ForeshadowViaFuzz, ShellShockViaFuzzing, Log4ShellViaFuzzing} and fuzzing is even recommended by the authors to prevent similar exploits \cite{35ForeshadowFuzzRecom, 33KrackViaFuzz}.

While fuzzing is often used for finding bugs in general, there are even fuzzers that have a focus on catching security vulnerabilities specifically. Fuzzers like Yuwei Li et al. \cite{32V-Fuzz} do this with their Vulnerability-Oriented Evolutionary Fuzzing tool.

\section{Goals}
\label{intro:Goals}
With this thesis we aim to be able to compare multiple know techniques for automatically finding bugs in constraint programming languages. As this type of language differs from most used programming languages not all techniques may work equally well. On top of that some bugs could have a big impact and are not always clear that they occurred.
The techniques used will be a modified fuzzer originally used on SMT problems, a second will be apply satisfiable preserving changes to known inputs and test that the original input matches the known input. Our last technique will be taking advantage to the fact that CPMpy has a big library of solvers, which will be used to check that all the solvers agree on the solutions.

\section{Research questions}
\label{intro:RQ:RQ's}
As the title of the thesis already may have spoiled it, we are trying out multiple fuzzing techniques out on CPMpy, with the goal of finding which technique works well for this specific type of programming language. This in order to give a push to identify ways of automatically discovering (and maybe solving) new bugs in constrain programming languages. We put forward two regions of research questions this thesis will focus on.

\subsection{Problem statement}
\label{intro:ProblemStatment}
As described in at the introduction of  this chapter \ref{intro:intro}, bugs are practically unavoidable and always unwanted, especially when a user trusts a program to give a correct answer and it does not. With solvers surrounding constraint programming languages being executed more and more we would like to strongly avoid any bugs in the real world from arising. To this end it would be interesting to find bugs during development without much overhead, a modern approach would be the use of fuzzers. which we will try out on a constraint programming language.

\subsection{Main focus: fuzzing technique-focused}
\label{intro:RQ:MainFocus}
The first and our main focus will be comparing different fuzzing techniques, we are going to modify a successful SMT fuzzer STORM to the CPMpy language, which we will name CTORM for CPMpy STORM. Try differential testing between the multiple solvers and out last technique is the use of metamorphic testing. Resulting in the following questions: \newline
Research question 1: What fuzzing technique will find the most bugs? \newline 
Research question 2: What fuzzing technique will find the most critical bugs? \newline
Research question 3: What type of bugs will be found by each fuzzing technique? \newline
%\subsubsection{subfocuses}
%Research question 4: Which metamorphic transformation find the most (critical) bugs? \newline
%focused on: \\
%bool, int, list, array

%\subsection{Solver-focused}
%A second focus we have goes more towards the different solvers and the differences between them resulting in. \newline
%Research question 5: Which solver has the most (critical) bugs? \newline

\subsection{Classification-focused}
\label{intro:RQ:ClassificationFocused}
Our next and last focus will be on the classification of found bugs, giving us the following research questions. \newline
%To then end with focus three based around the classification of found bugs, giving us the following research questions. \newline
Research question 4: How many (critical) bugs can we find? \newline
Research question 5: What are the causes of the bugs? \newline
%Research question 8: What are the type of bugs found? \newline

\subsection{Not focused: efficiency and others}
\label{intro:RQ:NotFocused}
A keen reader may wonder why we do not focus on efficiency, this would result in more bugs being caught in a smaller timeframe. While perfectly valid to investigate we believe that discovering which techniques works best for CP' has a higher value than on top of the techniques investigated being able to run automatically. The efficiency in CP has already significant research and literature on how to optimize the solvers, which take the most amount execution time compared to the testers.

% all written in Python


\section{Modus operandi}
This thesis will create three techniques to detect bugs automatically for the constraint programming language, CPMpy. Starting from multiple examples we will extract all solvable models to then run each model through a technique to see if we can create a bug either from the original model or by modifying it. Afterwards, we will validate the found bugs and evaluate and evaluated the found bugs for each technique. To finally compare the results and see which techniques are able to find bugs in constraint programming languages.

\todo{optional: section, written out text what each chapter is about}


%\section{Conclusion} intro chapter doesn't need a conclusion.

%%% Local Variables: 
%%% mode: latex
%%% TeX-master: "thesis"
%%% End: 

%written in USA's English
\documentclass[master=cws,masteroption=se,english]{kulemt} % oneside
\setup{% Verwijder de "%" op de volgende lijn bij UTF-8 karakterencodering
  inputenc=utf8,
  title={The best master's thesis ever}, 
  author={ing. Ruben Kindt},
  promotor={Prof.\, dr.\ Tias Guns},
  assessor={ },
  assistant={Ir. Ignace Bleukx}
}
%Which techniques are able to automatically find bugs in constraint programming languages?
%How can bugs be detected in constraint programming languages?
%Lifting combinatorial fuzz testing for constraint prog


% Verwijder de "%" op de volgende lijn als je de kaft wil afdrukken
%\setup{coverpageonly}
% Verwijder de "%" op de volgende lijn als je enkel de eerste pagina's wil
% afdrukken en de rest bv. via Word aanmaken.
%\setup{frontpagesonly}

% Kies de fonts voor de gewone tekst, bv. Latin Modern
\setup{font=lm}

% Hier kun je dan nog andere pakketten laden of eigen definities voorzien
\usepackage{todonotes}
\usepackage{pifont}   % used for checkmarks
\usepackage{listings} % used for code listings
\usepackage{csquotes} % makes sure that the quotes follow the same typeset as the rest of the text
\usepackage[tableposition=top]{caption} % caption above table
%\usepackage{float}

%thanks to https://github.com/youngkd/MSPSP-InstLib/blob/master/format-description.tex
%----------------------------------------------%
% Syntax highlighting for MiniZinc in listings %
%----------------------------------------------%
\definecolor{ForestGreen}{RGB}{34,106,46}
\usepackage{color}
\lstdefinelanguage{minizinc}{
	morekeywords={
		%% MiniZinc keywords
		%%
		ann, annotation, any, array, assert,
		bool,
		constraint,
		else, elseif, endif, enum, exists,
		float, forall, function,
		if, in, include, int,
		list,
		minimize, maximize,
		of, op, output,
		par, predicate,
		record,
		set, solve, string,
		test, then, tuple, type,
		var,
		where,
		%% MiniZinc functions
		%%
		abort, abs, acosh, array_intersect, array_union,
		array1d, array2d, array3d, array4d, array5d, array6d, asin, assert, atan,
		bool2int,
		card, ceil, combinator, concat, cos, cosh,
		dom, dom_array, dom_size, dominance,
		exp,
		fix, floor,
		index_set, index_set_1of2, index_set_2of2, index_set_1of3, index_set_2of3, index_set_3of3,
		int2float, is_fixed,
		join,
		lb, lb_array, length, let, ln, log, log2, log10,
		min, max,
		pow, product,
		round,
		set2array, show, show_int, show_float, sin, sinh, sqrt, sum,
		tan, tanh, trace,
		ub, and ub_array,
		%% Search keywords
		%%
		bool_search, int_search, seq_search, priority_search,
		%% MiniSearch keywords
		%%
		minisearch, search, while, repeat, next, commit, print, post, sol, scope, time_limit, break, fail
	},
	sensitive=true, % are the keywords case sensitive
	morecomment=[l][\em\color{ForestGreen}]{\%},
	%morecomment=[s]{/*}{*/},
	morestring=[b]",
}
%% Settings for listings
%%
\lstset{ %
	backgroundcolor=\color{lightgray},  % choose the background color; you must add
	% \usepackage{color} or \usepackage{xcolor}
	basicstyle=\scriptsize\ttfamily,    % the size of the fonts that are used for the code
	belowskip=0em,
	breakatwhitespace=false,            % sets if automatic breaks should only happen at whitespace
	breaklines=true,                    % sets automatic line breaking
	captionpos=b,                       % sets the caption-position to bottom
	commentstyle=\color{ForestGreen},   % comment style
	%deletekeywords={...},              % if you want to delete keywords from the given language
	escapeinside={\%*}{*)},             % if you want to add LaTeX within your code
	extendedchars=true,                 % lets you use non-ASCII characters; for 8-bits
	% encodings only, does not work with UTF-8
	frame=single,                       % adds a frame around the code
	keepspaces=true,                    % keeps spaces in text, useful for keeping indentation
	% of code (possibly needs columns=flexible)
	keywordstyle=\bfseries\color{blue}, % keyword style
	language=minizinc,                  % the language of the code
	%morekeywords={*,...},              % if you want to add more keywords to the set
	numbers=left,                       % where to put the line-numbers; possible values are (none, left, right)
	%numbersep=5pt,                     % how far the line-numbers are from the code
	%numberstyle=\tiny\color{Gray},     % the style that is used for the line-numbers
	rulecolor=\color{black},            % if not set, the frame-color may be changed
	% on line-breaks within not-black text (e.g. comments (green here))
	showspaces=false,                   % show spaces everywhere adding particular
	% underscores; it overrides 'showstringspaces'
	showstringspaces=false,             % underline spaces within strings only
	showtabs=false,                     % show tabs within strings adding particular underscores
	%stepnumber=1,                      % the step between two line-numbers. If it's 1, each line will be numbered
	stringstyle=\color{purple},            % string literal style
	tabsize=2,                          % sets default tabsize to 2 spaces
	title=\lstname                      % show the filename of files included with \lstinputlisting;
	% also try caption instead of title
}

%\usepackage[style=numeric,sortcites,sorting=nty,backref,hyperref]{biblatex}
%\usepackage[style=numeric,sortcites,sorting=none,backref,hyperref]{biblatex} % no sort, for finding unused ref
%\nocite{*}  % cite all ref, for finding unused ref
\usepackage{biblatex}
\bibliography{references}

% Tenslotte wordt hyperref gebruikt voor pdf bestanden.
% Dit mag verwijderd worden voor de af te drukken versie.
\usepackage[pdfusetitle,colorlinks,plainpages=false]{hyperref}
\hypersetup{allcolors = black}
%%%%%%%
% Om wat tekst te genereren wordt hier het lipsum pakket gebruikt.
% Bij een echte masterproef heb je dit natuurlijk nooit nodig!
%\IfFileExists{lipsum.sty}%
% {\usepackage{lipsum}\setlipsumdefault{11-13}}%
% {\newcommand{\lipsum}[1][11-13]{\par Hier komt wat tekst: lipsum ##1.\par}}
%%%%%%%
\begin{document}
\todo{Thesis Title, does not have to be a '?'}
\begin{preface}
  \todo{turn this in to sentence}
  professor Guns for the guidance % zou ik durven om die onder Ignace te zetten?
  Ignace Bleukx for answering the many questions, the intensive thesis meetings, proofreading and the cleverness for coming up with CTORM
  Jo Devriendt for finding bugs within our bug finder
  the many other CPMpy bug solvers
  friends for proofreading 
  and family for the support during my studies 
  
\end{preface}

\todo{remove todo list}
\listoftodos
\tableofcontents*
\setcounter{tocdepth}{5} % this works wow! Now I have a clean TOC but still the pdf outline in depth

\begin{abstract}
	\todo{abstract}
  Abstract 
  %The \texttt{abstract} environment contains a more extensive overview of
  %the work. But it should be limited to one page.
\end{abstract}

\begin{abstract*}
	\todo{samenvatting}

State of the art 
'in deze masterproef stellen wij een vergelijkende studie voor om auto bugs te vinden in cpl met als voorbeeld taal de CPMpy written by Tia's Guns at al.
Voordelen automatisch bugs (Tijdwinst, Safer programma)


  %In dit \texttt{abstract} environment wordt een al dan niet uitgebreide
  %Nederlandse samenvatting van het werk gegeven.
  %Wanneer de tekst voor een Nederlandstalige master in het Engels wordt
  %geschreven, wordt hier normaal een uitgebreide samenvatting verwacht,
  %bijvoorbeeld een tiental bladzijden. 
\end{abstract*}

% Een lijst van figuren en tabellen is optioneel
%\listoffigures
%\listoftables
% Bij een beperkt aantal figuren en tabellen gebruik je liever het volgende:
\listoffiguresandtables
\lstlistoflistings
% De lijst van symbolen is eveneens optioneel.
% Deze lijst moet wel manueel aangemaakt worden, bv. als volgt:
\chapter{List of Abbreviations and Symbols}
\section*{Abbreviations}
\begin{flushleft}
  \renewcommand{\arraystretch}{1.1}
  \begin{tabularx}{\textwidth}{@{}p{14mm}X@{}}
  	
%Alphabetic pls
	CI/CD & Continuous Integration and Continuous Deployment, a pipeline for newly written code to repeatably be build, test, release, deploy and more. \\
    CP & Constrain Programming language sometimes also referred to as CPL \\
    CPL & Constrain Programming Language also referred to as CP \\
    CNF & Conjunctive Normal Form, which is a boolean formula written using conjunctions of distinctions. \\
    CSP & Constraint Satisfaction Problem is a problem with constraints and variable with a specific domain e.g., boolean, finite and others.\\
    CPMpy & Constraint Programming and Modeling language for Python.\\
    CVC & Cooperating Validity Checker a popular SMT-theorem prover \\
    PUT & Program Under Test, the piece of code, application of program that is tested on for potential bugs. \\
    LLVM & Although it looks like an abbreviation, it is not. LLVM is the name of a project focused on compiler and toolchain technologies. \\
    MIP & Mixed Integer Programming, theory where decision variables are allowed to be integers \\
    MUS & Minimal Unsatisfiable Subset, the smallest subset possible that is not satisfiable \\
    SMT & Satisfiability Modulo Theory \\
    SUT & Software under Test, analogue to PUT \\
  \end{tabularx}
\end{flushleft}

\section*{Symbols}
\begin{flushleft}
  \renewcommand{\arraystretch}{1.1}
  \begin{tabularx}{\textwidth}{@{}p{14mm}X@{}}
 	$\sim$ & negation used by CPMpy\\
 	$\&$ & and used by CPMpy\\
 	$\vert$ & or used by CPMpy\\
  	$\neg$ & logical negation \\
  	$\land$ & logical and \\
  	$\lor$  & logical or \\
  \end{tabularx}
\end{flushleft}

% Nu begint de eigenlijke tekst
\mainmatter

\chapter{Introduction}
\label{cha:intro}
Programmers make mistakes, just like everyone.
Software is often complex, written by multiple people, sometimes used with the wrong assumptions or does not meet the objective goals. 
There are a lot of causes for bugs: software complexity, multiple people writing different parts of software, changing objective goals of software, misaligned assumptions and more. Most these things can not be avoided during the creation of software but do cause program crashes, vulnerabilities, wrong outcomes and more.
Multiple forms of prevention have been created in various forms of software testers, documentation, automatic tests, code reviews. All of these aim to prevent the occurrence of bugs. While automatic test cases often evaluate goals of software end evaluate previous known bugs, it can do much more.
Fuzzing software is one of those things, a technique that is popular in the security world for exploit prevention, which generates random input for a program under test (PUT) and monitors if the program crashes or not. This explanation was the original interpretation of fuzzing as preformed by \cite{originalFuzzingUnixUtils}


The first contains a general introduction to the work. The goals are\todo{remove this} defined and the modus operandi is explained.

%%% Local Variables: 
%%% mode: latex
%%% TeX-master: "thesis"
%%% End: 

\chapter{Fuzzing}
\label{cha:2:fuzzing}
A chapter is a logical unit. It normally starts with an introduction, which
you are reading now. The last topic of the chapter holds the conclusion.
\todo{intro chapter 1}

%if inf seed set, reducing it is better with the following technique (the min seed subset without any code coverage loss) \cite{14rebert2014seedselecting}
%seed selection problem, inf seeds which one do we use
%other fuzzers internal workings
%Chapter around fuzzing
%	\cite{mathis2019parser} states 3 optins traditional, stochastic and syntax driven.
%avoiding inf loops -> timeouts
%
%subsection about (psuedo-)randomness


\section{History} \todo{fix title}
The rise of fuzzing came when Miller gave a classroom assignment\cite{21FuzzingAssignment} in 1988 to his computer science students to test Unix utilities with randomly generated inputs with the goal to break the utilities. Two years later in December he wrote a paper\cite{4originalFuzzingUnixUtils} about the remarkable results. That more than 24\% to 33\% of the programs crashed.
In the last thirty years the technique of fuzzing has changed significantly and various classifications have come forward\cite{12Fuzzingasurvey}\cite{13manes2019survey}. Three most used classifications are: 
how does the fuzzer create input, how well is the input structured and does the fuzzer have knowledge of the program under test (PUT)?


%generation or mutatuion
%black, gray white box
%lexical, sematic, constraint, random

\section{Generation of mutation}
A fuzzer can construct input for a PUT in two ways, it can generate input itself or it can take an existing input and modify it, which are often called seeds. While Generation is more common than modifying when it comes to in smaller inputs the opposite is true for larger inputs. This is cause by the fact that generating semi-valid input becomes a lot harder the longer the input becomes. For example generating the word "Fuzzing" by uniformly random sampling using ASCII, has a chance of one in $5*10^{14}$ of happening, making this technique infeasible when we want to generate bigger semi-valid inputs. With mutation we can start of with larger and valid input and make modifications to create semi-valid inputs. With this last technique the diversity of the seeding inputs does become quite important. Ideally we would have an unlimited diverse set of inputs, but due to limited computation and available inputs we sometimes need to take a subset. In a paper by Rebert et al. \cite{14rebert2014seedselecting} they say that seed selection algorithms can help and compare random seed selection to the minimal subset of seeds with the highest code coverage among other seed selection algorithms. 

\section{Input structure}
%lexical, sementical, constraint or random
While we have discussed the bigger scope on how inputs are created, let us go into more detail. As we have seen before fuzzing started mainly with Miller's assignment, that random generation of inputs falls under 'dumb' fuzzing. Due to only seeing the input as one long list of strings with no knowledge of any substrings. This technique can be applied to mutation as well, compared to only adding with generation here we also additionally remove or change randomly selected symbols of the seed with new symbols. 
We can create three types of inputs: valid, semi-valid and nonsense input. With nonsense input we will be testing the parser almost exclusively, either it crashes the parser or the parser will say invalid and the PUT will stop running. With semi-valid input we hope to be as close as possible to valid input to be able to explore the PUT but to catch an edge cases and crash.

A smarter technique is referred to one which has knowledge about the structure inputs can have or should have. This increases the chance of inputs passing the parser, being able to test the deeper parts of the PUT and as such covering more of the PUT. This with an increased complex fuzzer. We can build a 'smart' fuzzer by adding knowledge about keywords, making it a lexical fuzzer, adding knowledge about syntax, for example all parentheses needs to be matched. Directed fuzz testing does fit in this category as well but is only possible in a white box environment, more on that later.

\section{Black, grey or white box fuzzing}
Now that we have discussed adding knowledge of inputs to the fuzzer, we can also add knowledge about the PUT to the fuzzer. Which brings us to black, grey and white box fuzzing. With black box fuzzing we have no knowledge about the working of the PUT and we treat the PUT as a literal black box, we present out input and we look at what comes out of it. With only minimal information the fuzzer then tries to improve its input creation. This was also the technique that Miller (unknowingly) used.
 
With grey box fuzzing is usually comes with tools that give indirect information to the fuzzer, tools like: code coverage measurements, timings, types of errors and more.

Then we arrived at white box testing, with this technique fuzzers can know the source code and can adjust there inputs to fuzz specific parts of the code at a higher cost due to having to reverse engineer the path to specific edge cases. You may may have already suspected, white box fuzzing has more knowledge and can find more bugs per input but creating those inputs take more time compared to black box fuzzing.

The differentiation between black, grey and white box fuzzing is not clear cut, most people would agree that white box fuzzing has full knowledge about the PUT, including the source code, that grey box fuzzing has some knowledge about the PUT and that black box fuzzing has little to no knowledge about the PUT. Going into more detail all we can say is that it is no longer a black-and-white situation and that the lines become more fuzzy. \todo{is this pun allowed?}


\section{Fuzzing small programs} \todo{fix title}
\section{Fuzzing big programs}

\subsection{clasical fuzzing (for some original cite's)}
\subsection{AFL++}
\subsection{KLEE}
\subsection{ClusterFuzz}
\subsection{STORM (startingpaper)}
A novel way of approaching this oracle problem is by Alexandra Bugariu and Peter M\"uller in "Automatically testing string solvers"\cite{9bugariu2020automaticallyTestingStringSolvers} where they know the (un)satisfiability of the formulas by the way of construction, called (un)sat by construction.  For formulas that satisfy they generate trivial satisfiable formulas and then by satisfiability preserving transformations create more complex formulas. And unsatisfiable formulas they use $\neg$ A $\land$ A', with A' being a equivalent formula of A, to create the trivial unsatisfiable formulas. To increase the complexity of those trivial formulas, they again depend on satisfiability preserving transformation.
This technique has also been applied to SMT solvers by Muhammad Numair Mansur at al.\cite{1mansur2020detecting}

\subsection{types of bugs} \cite{1mansur2020detecting}

\subsection{oracle (what is a crash)}
%oracle problem? see holy grail
%somewhere a ref to later chapter input simplification (minimisation differantioation)

\section{Conclusion}
The final section of the chapter gives an overview of the important results
of this chapter. This implies that the introductory chapter and the
concluding chapter don't need a conclusion.


%%% Local Variables: 
%%% mode: latex
%%% TeX-master: "thesis"
%%% End: 
 % CP, SAT and SMT 
\chapter{Detecting crucial parts of the input Simplifying Inputs} \todo{better title}
\label{cha:3}
When we detect that the PUT crashes, wrongly satisfies, wrongly not satisfies or hangs on a given input we now want to know why it does that. What causes this unwanted output and what line the bug occurs. With crashes, a stack trace and some luck this could be easy, but when the crash is not main perpetrator or we get an other unwanted output the developer my need to debug deep into the code to find the bug. This with a potential large input could be a tedious and long assignment, for this reason we would like to know what parts of the input are related to the bug. We will discover this further in this chapter, starting with \todo{plug in sections}

\section{Minimizing inputs}
As said with bigger inputs it takes longer to find the bugs due to having to find identify which parts of the input trigger the error or having to walk through the execution of the PUT if even possible. This is where techniques come into play to create detect the crucial parts of the input. This while still finding the same bug and not changing a null pointer dereference to a parser related bug for example.\cite{bookZellerwhyProgramsFail}

\subsection{Simplifying}
The first and the most used technique is the simplification of the inputs where we remove parts of a failing input and check if it still fails. When no longer possible to remove any part of the input we have obtained an input where all parts are needed to expose the bug. This input is at the same time also the shortest possible input to trigger this bug, making finding the bug easier than in the original input filled with unrelated parts. 
\todo{add mozilla ref}
\subsection{Isolation}
Another technique, isolation, is explained in Andreas Zeller et al. 
\cite{5zeller2002simplifyingIsolatingFailure-inducing} this is a technique where instead of minimizing the input we try to find the smallest difference between an input that shows the bug versus an input that does not show it. This with the advantage that no matter the if we find the bug or not the difference will diminish, either the maximum input will shrink or the minimum input will grow. Although we will have to make sure that we are still finding the same bug and not moving from a null pointer dereference to a parser related bug. Isolation also brings extra complexity with the tracking of multiple inputs and the maximal input could take longer to run due to its size, but according to Andreas Zeller et al. is the faster one .

% what size to remove, importent for speed
% what are both, what situations is which the better one?  donno atm
%	Isolation will require a lot more things to track but is faster
\cite{bookZellerwhyProgramsFail}
%	isolation vs simplifying \cite{zeller2009programs} p 285
%
%The Precision Effect minimizing/simplyfying may lead to a dif found bug if multiple, since all need to be solved not problem  \cite{zeller2002simplifying}
%	 can be solved with stack trace comparison
\subsection{Small inputs by creation}
small by construction source 9

\todo{algorithms?} paper 5 has nice simplification to isolation delta deb, called 'ddmin()'

\section{Delta-debugging}
\cite{2FuzzingAndDelta-debuggingSMTSolvers}


%chapter on simplifying the crashes
%	binary search will not work all the time
%	quarters remove may work (if all parts fail go more granular, 1/9 or 1/16)
%		start with halfs then *2
%		always search further with same granularity but with removed part until all options with that granularity searched \cite{zeller2009programs} p111
%		this uses no knowledge from input structure and program structure \cite{zeller2009programs} p112
%	delta debugging
%		time spend searching vs simplified ratio is important as mentioned in \cite{mansur2020detecting}
%		and needs to preserve satisfiability as mentioned in \cite{mansur2020detecting}
%		^ possibly a big deal to find critical bugs
%
%	with knowledge of input, syntax \cite{zeller2009programs}
%	of by bigger entities like lines of words \cite{zeller2009programs}
%	 for speed
%
%	alt approach like \cite{mansur2020detecting} try finding the bug again with less resources avail
%	or isolaytion \cite{zeller2009programs} p 285 
%		I think it may fail if multiple parts are relevant
%		I think it could detect for example the CPMpy import as a bug cause as the min diff that causes the bug
%
%	sub section on MUS/minimum unsat subset vs delta debugging
%		MUs good for only whole constraints while 
%		delta debugging goes for partial structures
%

\section{Deduplication}
Another thing to notice is that multiple inputs could prompt the same bug from occurring, these inputs could be similar but don't have to be. With simplifying the input we should be able to detect exact copies, but depending on the simplification's time complexity other techniques could be better with similar results. In case where we would have access to stack traces (via crashes or hanging PUT's) we could differentiate the bugs on basis of the hash of multiple lines from the backtrace sometimes even numerous hashes per input. this technique is called stack backtrace hashing and is quite popular according to Valentin J.M. Man\`es et al.\cite{13manes2019survey}. Another technique talked about in that paper, is looking at the code coverage generated by the inputs where we use the executed path (or hash of it) is used as a fingerprint of the inputs. A technique, used by Microsoft\cite{36semanticsAwareDeduplicationRETracer} is called semantics based deduplication, where in stead of back track use memory dumps to hopefully find the origins of bugs. This use of dumps is less ideal due to traces having more information, but the latter is not always possible due to the performance overhead and privacy causes as specified in the paper. A last technique would be looking at the bug description left by a manual bug reports by the user, although this dependence on the quality of the bug reports and is most likely poorly automatable. None of the techniques mentioned above are perfect: with stack backtrace hashing you could find to many false positives or false negatives depending on the depth taken from the stack, with coverage some inputs will generate extra function calls and the semantics based deduplication are limited to X86 or x86-64 code with the binary file and the debug information. Neither of these techniques work with black box fuzzing unfortunately.

\section{Conclusion}
The final section of the chapter gives an overview of the important results
of this chapter. This implies that the introductory chapter and the
concluding chapter don't need a conclusion.


%%% Local Variables: 
%%% mode: latex
%%% TeX-master: "thesis"
%%% End: 
 % Fuzzing
\chapter{CCPMpy}
\label{cha:4}
\todo{intro ch 4}

\section{The First Topic of this Chapter}

% see holy grail
%
%\section{Figures}
%Figures are used to add illustrations to the text. The \fref{fig:logo} shows
%the KU~Leuven logo as an illustration.
%\begin{figure}
%	\centering
%	\includegraphics{logokul}
%	\caption{The KU~Leuven logo.}
%	\label{fig:logo}
%\end{figure}
%
%\section{Tables}
%Tables are used to present data neatly arranged. A table is normally
%not a spreadsheet! Compare \tref{tab:wrong} en \tref{tab:ok}: which table do
%you prefer?
%
%\begin{table}
%	\centering
%	\begin{tabular}{||l|lr||} \hline
%		gnats     & gram      & \$13.65 \\ \cline{2-3}
%		& each      & .01 \\ \hline
%		gnu       & stuffed   & 92.50 \\ \cline{1-1} \cline{3-3}
%		emu       &           & 33.33 \\ \hline
%		armadillo & frozen    & 8.99 \\ \hline
%	\end{tabular}
%	\caption{A table with the wrong layout.}
%	\label{tab:wrong}
%\end{table}
%
%\begin{table}
%	\centering
%	\begin{tabular}{@{}llr@{}} \toprule
%		\multicolumn{2}{c}{Item} \\ \cmidrule(r){1-2}
%		Animal    & Description & Price (\$)\\ \midrule
%		Gnat      & per gram    & 13.65 \\
%		& each        & 0.01 \\
%		Gnu       & stuffed     & 92.50 \\
%		Emu       & stuffed     & 33.33 \\
%		Armadillo & frozen      & 8.99 \\ \bottomrule
%	\end{tabular}
%	\caption{A table with the correct layout.}
%	\label{tab:ok}
%\end{table}


\section{Conclusion}
The final section of the chapter gives an overview of the important results
of this chapter. This implies that the introductory chapter and the
concluding chapter don't need a conclusion.

%%% Local Variables: 
%%% mode: latex
%%% TeX-master: "thesis"
%%% End: 
 % Detecting crucial parts in inputs
%\include{chap-5RQ} % Research questions
\chapter{Implementation}
\label{cha:5:impl}
\label{impl:Intro}
In this chapter we will discuss how we build our fuzzers, what issues we had to circumvent and how we did that. Starting off with how we got our seeds to fuzz upon, to then discuss how we implemented the three techniques to finally end with how we deobfuscated the found bugs.

\section{Software versions used}
\label{impl:softwareVersion}
Throughout this paper we used CPMpy\footnote{\url{https://github.com/CPMpy/cpmpy}} version V0.9.9 (commit \href{https://github.com/CPMpy/cpmpy/commit/e79b3afedc934a9437c2ddb3a9f54d7e2d7bd3ee}{e79b3af}), unless specified otherwise, this version was chosen as it was the latest release version at the time of testing the first technique. All techniques were developed in Python 3.8, the MiniZinc solvers came with MiniZinc Python\footnote{\url{https://github.com/MiniZinc/minizinc-python}} release version 0.7.0 (commit \href{https://github.com/MiniZinc/minizinc-python/commit/a195cf63fcfbc98665d70ab64efb5424db25bd7e}{a195cf6}). For the proprietary solver Gurobi\footnote{\url{https://www.gurobi.com/}}, we used its Python version 9.5.2 with an academic license. 
%problems Gurobi outputting Not_run this was a trial problem, with the academic version does not occure
Originally, we did try to utilize the trial version to ease possible reproducibility, but the restrictions on the complexity of the problems became a hindrance which resulted in us moving to an academic license. For the other versions of the solvers, we used the ones included in the already mentioned packages, except for MiniZinc’s transformations to Google’s OR-Tools\footnote{\url{https://github.com/google/or-tools}}, there we had to install OR-Tools for MiniZinc manually, which we did using release version 9.3.10497 (commit \href{https://github.com/google/or-tools/commit/49b6301e1e1e231d654d79b6032e79809868a70e}{49b6301}).


\section{Obtaining seeds}
\label{impl:obtainingSeeds}
As discussed in a previous section (Section \ref{fuzzing:generationMutation}) generating new inputs is significantly harder than mutation, but with the latter one we require a diverse set of seed files. Fortunately, the CPMpy team made a lot of documentation and examples on how to model problems in their language. Ranging from easy examples to teach the language to advanced examples in order to showcase certain features. 
At the moment of writing most examples are found in the main branch and some extras can be found on the “csplib” branch\footnote{\url{https://github.com/CPMpy/cpmpy/tree/csplib}} waiting to be merged with the main branch. We downloaded a copy of those branches on Tuesday 27th of September to be used as future seed files. 

A second source of seeds files came from Hakan Kjellerstrand a retired software developer and independent researcher from Sweden which was found while reading “Model-Based Algorithm Configuration with Adaptive Capping and Prior Distributions” \cite{18bleukx2022model}. Mr. Kjellerstrand has a big repository\footnote{\url{https://github.com/hakank/hakank/tree/master/cpmpy}} full of problem models which he solves in multiple ways, including CPMpy. We obtained a copy of all his CPMpy examples on Tuesday 27th of September to top off our collection of future seed files.


After that we ran all seeds to test that the non-modified seeds do not crash on their own and noticed that most examples ran in less than a minute. The handful of examples that did run long were left out or were simplified to gain a speed up while solving them. Knowing that all seeds are capable of being run in a minute helps us avoid the halting problem. If a modified seed starts running significantly longer than a minute we can start investigating that seed for a potential bug. A final change we made to the future seed files is extracting the model from each file found on the repositories. We did this for a couple of reasons: some files had a loop around the solve instruction combined with small changes or had multiple problems in one file, this gave us a separate model for each found problem. In order to extract these constraints, we temporarily modified CPMpy to extract the created model, variables and constraints included, each time solve was called, this resulted in over nine thousand problem models which we will use as our seed files.


%We extracted our seeds twice, a first time where we extract our model without any flattening of the constraints and a second time where we did flatten the constraints. While building up a model of the problem CPMpy allows for arbitrary complex compositions of constraints resulting in a nested tree of constraints. However, not all solvers allow this nested tree as described by the documentations of CPMpy. It is for that reason that CPMpy flattens the constraints to what they call ‘normal forms’ as the similar definition of SAT but with a disclaimer that this does not exits to their knowledge with which we agree with, with this flattened form CPMpy can directly call the solvers or do some changes for the solver interface on the flattened constraints to then send it to the solver. The reason we extracted our seeds with and without a flattening process is that \todo{aanvullen na vraag} a flattened version and the reason we did is without will become clear in the next section.

\section{Modifying STORM into CTORM}
\label{impl:modifyingSTROM}
Our first technique of finding bugs is heavily based on STORM which we shortly discussed before in Section \ref{fuzzing:testingWithFuzzers}. Instead of searching for SMT bugs, here we want to be able to find bugs in constraint programming languages and specifically in CPMpy. We downloaded STORM from the repository\footnote{\url{https://github.com/Practical-Formal-Methods/storm}} on Tuesday 27th September.
%https://github.com/Practical-Formal-Methods/storm/commit/55d091624523a0544112ffc339fe81103b3daa2b
The original plan was to convert our seeds to FlatZinc using the MiniZinc API provided by CPMpy to then convert that to SMT-LIB \cite{72bofill2010system} using Miquel Bofill et al.’s fzn2smt-tool to then be able to use STORM as it was built originally. Unfortunately (and a bit predictable), this way of working did not work out. On top of fzn2smt being more than a decade old, the multiple transformation layers that could introduce conversion bugs and the unclear way back from SMT-LIB prevented this path from being investigated by us.

Therefore, we decided to refactor STORM to fit CPMpy and name the technique CTORM for CPMpy-STORM. To change STORM into CTORM, we needed to rewrite the detection, labeling and construction of (sub)constraints, this refactoring did come with some downsides, some features of STORM no longer work such as incremental solving or the input obfuscation that was built-in. A bigger downside came with the refactoring of the negation function of STORM, as CPMpy is still in active development and 
the negation not always being implemented already, this was felt while trying to negate global constraints. I.e., when trying to invert (sub)constraints which include \texttt{alldifferent([var1, var2, var3])} using CPMpy, it crashed, this is of course a bug (more specifically not yet implemented) in CPMpy but used by the fuzzer. So here we had the choice of adding the missing negation of global constraints to CPMpy or to limit our fuzzer to not use the missing features. We choose to limit the fuzzer, since we are trying to detect bugs in CPMpy with different tools and extending the language ourselves goes out of scope of this thesis. 

%remove?
%The resulting limitation on our fuzzer only influences the speed of generating new constraints and it can theoretically now get stuck but this has not happened yet, so we believe it to be a acceptable limitation.


We gave only non-flattened inputs to this solver since both STORM and CTORM used a recursive process to get all subformulas because we wanted to change as little as possible to the inner workings of CTORM compared to STORM. The flattening process in CPMpy is used to convert the potential tree-like constraint structures to a flattened list of constraints which the solver can handle. 
In order to get those non-flattened inputs, we hijacked the flatten process of CPMpy to return all subformulas before returning the flattened constraints, this gave access to the more convoluted subformulas to use in the next steps of CTORM. This flattening process was done before any modifications were made, so in the eyes of the fuzzer it got flattened seeds but with the knowledge of some more complex constraints just like STORM and CTORM does. For each input CTORM combines 100 new constraints built from the existing constraints, this is repeated to create a hundred models to then check if the result matched with the original output in CPMpy.
%\todo{optional image of this CTORM processes?}

%DUMB ideas:
%Translating seeds from solver X to solver Y 
%option 1 hardcode default solver of CPMpy to Y, less good modifying the language is something we want to avoid. May also not work when solver is hard coded in the seed.
%Option 2 interpret the seed and make changes so that the solver Y is run. Bit trickier as you cannot see the difference between model.solve() and solver.solve() because model and solver are variables.
%
%nested functions with global function inside are giving problems


\section{Metamorphic testing}
\label{impl:Meta}
While CTORM was quite autonomous, metamorphic testing did take one step back to manual work, as this technique requires some metamorphic transformations, these transformations take a (or multiple) constraint(s) of our seed problems and change them repeatedly while keeping the (un)satisfiability the same. To then test if the original seed problem gives the same result as our modified problem, as discussed in Subsection \ref{fuzzing:MetamorphicTesting}. 

With papers such as \cite{50akgun2018metamorphic, 49usman2020testmc, 43YinYang} and others giving us inspiration, we came up with but not limited to the following 30 metamorphic relations. Replacing global constraints like \texttt{alldifferent([var1, var2, var3])} to their decomposition \texttt{var1!=var2 and var1!=var3 and var2!=var3}. Adding futile variables to global constraints such as \texttt{allequal([var4, var5])} by copying variables which did not limit (or restrict) the existing solution-space. We did this too for other more basic operations such as “and”, “or”, “xor”, “->” (implication), all forms of comparisons, min, max and others. We also included metamorphic relations proposed by the authors of “Validating SMT Solvers via Semantic Fusion” \cite{43YinYang}, those being semantic fusion for addition, subtraction, multiplication, and, or, xor and the comparisons. All analog to the example given in Subsection \ref{fuzzing:SemanticFusion}. 

We also linked multiple (sub)constraints of the problem to each other and replaced comparisons by other equivalent comparisons. Lastly, we also added new constraints which were independent of the original problem only to get in the way of the seed problem or be used in other metamorphic relations.

All these metamorphic relations individually were quite simple and should be handled easily by the flattening process, other CPMpy processes or by the solvers, but by combining multiple relations at random we were able to create more complex constraints that were not always handled correctly. Finally, we should note that while finishing this thesis Jo Devriendt found a bug within the metamorphic tester that made cyclic expressions possible which will crash CPMpy, which was swiftly fixed in the tester. More information about this bug can be found in the CPMpy bug report number \href{https://github.com/CPMpy/cpmpy/issues/163}{163}.

\section{Differential testing}
\label{impl:diff}
With differential testing we stepped a bit further away from the fuzzing world since we did not edit the input files.
With this last technique we put the (sub)solvers against each other, if solver A said that the problem is unsatisfiable and all other solvers said that the same problem was satisfiable we can say that we found a bug and that the bug was most likely to be in solver A, which differs from the previous techniques where we knew the correct solution in advance.

%This test could easily be integrated in one (or both) previous fuzzers 
% since this can be integrated as an extra check in the fuzzers. 

The way this tester was written is quite simple, we tested a given input on multiple solvers and searched if there was any difference in outputs or on the number of solutions provided in the results.
%ony 2 solvers
However, we discovered that only two solvers, namely “ortools” and “gurobi”-solvers, are able to search all solutions for problem models that contain global constraints. More solvers are available to find all solutions for SAT problems but 
the main objective of this thesis is to find CP-related bugs. The limitation to only two solvers results in only being able to compare two different implementations and has a risk of overlapping bugs. Preferably, we would have three or more solvers to be able to compare between them and also be able to automatically show us which of the solvers is likely to be wrong. In the future more solvers will be available within CPMpy with the capability of finding all solutions, but right now these tests are performed with two solvers. Small note, a limit of 100 solutions was put in place, otherwise finding solutions would take an unreasonable amount of time. 

When we look at only searching for one solution for a given example, we had more than enough solvers to compare between. Most of the times 14 solvers were compared and we never got lower than 6 solvers, this variation is due to the features used in the given problem the amount of solvers changed. If a problem happens to not use any global constraints the SAT-solvers may be able to solve them as well allowing us to compare between up to 36 solvers. Given that we only search for a single solution we are limited in what we can compare. For example, if solver A says that it found a satisfiable problem with the solution having values A' and solver B says the same but with different values B' (with A $\neq$ B ) the only thing we can compare is that both solvers output the same “satisfiable”. Since both solution A' and B' can be different solutions to the same problem.


\section{Detecting the cause of the bug}
\label{impl:DetectingCause}
As described in Chapter \ref{inputReduction:intro} finding a bug is one part of the problem, the other is finding which part of the inputs causes the bug. Submitting a complex input would result in a lot of work for the development team in order to find the cause. To avoid this situation, we used a Minimal Unsatisfiable Subset finder created by the CPMpy-team. The first version we used can be found in the advanced folder\footnote{\url{https://github.com/CPMpy/cpmpy/blob/master/examples/advanced/musx.py}} of the examples of CPMpy version 0.9.9. It was limited to finding MUS with the OR-Tools solver only, which caused us to write our own programs to deobfuscate the inputs based on this MUS-finder. 

We created a program to simplify inputs as described in Subsection \ref{inputReduction:Simplifying} and a program to isolate inputs as detailed in Subsection \ref{inputReduction:Isolation}. Nevertheless, we ended up only using the simplification technique due to not wanting to report a crucial part of the bug in one issue and reporting the difference between two inputs in the next issue. The already mentioned time penalty was not noticeable due to the inputs not being enormous. Both the programs we created did not give a minimal input back but did minimize the number of constraints, the difference being that some constraints could contain non-crucial parts of the bug. For example, a constraint “[15][variable1](...)” would give the exact same bug as constraint “[0, 0, 0, 0, 0, 0, 0, 0, 0, ... 0, 15][variable1](...)” but be clearer for the developer. 

To achieve a minimal input, we were planning to update our previously created programs, but CPMpy version 0.9.10 came out and contained a better MUS-finding program\footnote{\url{https://github.com/CPMpy/cpmpy/blob/master/cpmpy/tools/mus.py}} suitable for finding MUS for all available solvers, which was important since not all (sub)solvers would agree on the problem being satisfiable or not.


Nevertheless letting a solver search for an unsatisfiable subset in what it thinks is a satisfiable problem does not work, as it will never find an unsatisfiable subset for obvious reasons.
It was a similar case for crashes, for which we used a modified version of our simplification program. For obvious reasons the MUS-finders of CPMpy did not work for crashing programs. Our simplification program would test if the model still had a crash with some missing constraints or not and continue analog to the MUS version. It would again result in a minimal number of constraints but not in a minimal model. To get a minimal model some manual work was needed but that was not a significant amount of work. 

%\section{Tested solver}
%As a small side note not all solver within CPMpy are that usseefull if we wnat to find bug in CP languages 
%\todo{focused solvers}
%pySSD + PYsat: graph maker minder relevant omdat we CP fouten willen vinden 


\section{Conclusion}
\label{impl:conclusion}
In this chapter we discussed which software was used in tandem with the version’s numbers and GitHub repositories where applicable. We showed how we turned CPMpy’s and Hakan’s examples into the seed files we then used for our three techniques. After that, we specified how we implemented our techniques to find bugs, this being the modification of STORM into CTORM, the creation of metamorphic relations and the comparison of the output of (sub)solvers in the differential testing. To finally end with the tools used to minimize the input files once a bug was found.


%%% Local Variables: 
%%% mode: latex
%%% TeX-master: "thesis"
%%% End: 
 % Implementation
\chapter{Results}
\label{cha:6}
\todo{intro ch 6}

%serverspecs
%time when faults are found?

\section{STORM}
Techniques are best applied during development, since the modified storm of to often detects the already fond bugs

%problems limited by global fucntions of used seeds (somewhat fine),
%only 'and' and 'not' combinations Just like STORM

~(~(True))
~(global function)
\section{Differential testing}
solverlookup()  was during development = manual testing

\section{Metamorphic testing}
did not find ~(~()) simly because we dind't think to check it specificly, we did check ~=0
all checks ? + combo-able

\section{YinYang}


\section{unsat}
due to the way of importing the file a lot of edge problems become sat or unsat

\section{woringly sat}
meta unsat minizinc 0
to shirk this we used the fact that ortools+gurobi (probably) gave the correct solution, this bieing unsat and taking the mus from that


\subsection{finding mus}
combination of theire tool, but when we got errors from the different way of loading, We used my tool en then theirs. general combinations were done got get a minimal


\section{title}
%these tools often gave the same error again causing the new bug to be hidden, se choice to catch some of the more prevelent bugs, If some one wishes to rerun (parts) It is best to remove those exceptions in order to see the full output


\section{Conclusion}
\todo{Conclusion}
The final section of the chapter gives an overview of the important results
of this chapter. This implies that the introductory chapter and the
concluding chapter don't need a conclusion.

%%% Local Variables: 
%%% mode: latex
%%% TeX-master: "thesis"
%%% End: 
 % results
\chapter{Results}
\label{cha:res}
\label{res:Intro}
In this chapter we will see the results of all three of the created techniques, we will explain how we prevented frequently occurring bugs, discuss a diverse subset of found bugs with more detail. And end the chapter with a classification of all bugs and the reception of the bugs.

% multiple runs due toe randomness, how measured
% comparision with other fuzzers
% why did we (not) do code coverage ?
% timeout on the minizinc, but not on the ortools -> manuel work
% if stoppted before it was done due to timeout solvers didn't agree on the amount of solutions

\section{Running the tests}
\label{res:RunningTests}
\label{res:Specs}
Although, the specifications are not crucial since we did not do any speed benchmarking. It does give you, the reader, an idea of the performance. All tests were executed on an Ubuntu 20.04.5 LTS with 8GB RAM, an Intel core i5-3380M capable of 2.90GHz and a V-NAND SSD of 500GB 860 EVO Model MZ-76E500 through a SATA 3 connection. With each technique taking around a day to three days to run more than the nine thousand seed files once. Note that processing a seed once can mean that variants of that seed were run up to 100 times depending on the technique used.

%time when faults are found?

\subsection{Preventing the same bug from occurring frequently}
%these tools often gave the same error again causing the new bug to be hidden, se choice to catch some of the more prevelent bugs, If some one wishes to rerun (parts) It is best to remove those exceptions in order to see the full output
While the techniques were running, we noticed that once a bug was found by any of our techniques that same bug would occur frequently, but in a different seed causing our resulting logs to be cluttered with duplicates bugs. We were aware that this could occur and we originally wanted to use deduplication as described in subsection \ref{inputReduction:Deduplication} to get rid of the equivalent bugs. But after seeing 10 000 logs of the same bug, we changed from a reactive approach to a preventative approach. Once we noted a bug we tried to prevent it, for crashes this meant adding a try catch for the specific error. While for wrongly (un)satisfiable bugs we looked at special occurrences of keywords in the constraints we knew caused bugs, to then not log them. For example, a bug we will discuss soon had a specific string of characters, " == 0 == 0", making any problem wrongly unsatisfy. Being able to check the string of characters with knowing that it resulted in wrongly unsatisfiable solutions made it able for us to filter that bug as well. Luckily, with these restrictions on the output logs we were able to filter out most duplicate bugs, this is not an ideal solution but it worked well enough. The ideal solution would be that the fuzzer has knowledge of the already found bugs and reject them, techniques like these do exist but would bring us to far from the scope of the thesis. Although, less efficient the techniques could be used for detecting an error, fixing them and then rerunning for the next bug instead of adding exceptions.

\section{Results: found bugs}
\label{res:bugs}
%\todo{19 bugs is af h van welke definitie men volgt, met de onze komen we aan 19}
In total we found 19 bugs, three of which were already known in one form or another as an issue. Of those bugs some of them were easy fixes, some were a bit harder and required more time to solve. Depending on the definition of a bug the number of found bugs would differ with the definition followed in this paper 19 bugs remain out of the the 22 submitted. At the time of writing not al bugs are resolved (some are just reported days ago), but we look in anticipation how those will be solved. For now, let us look at some of the most interesting bugs we found. In order to do this, we will work with the four components we defined earlier in subsection \ref{CP:CPMpy}, this being: the model, the transformations, the solver interface and the solvers themselves as seen in figure \ref{fig:4ComponentsOfCPMpy}.


\subsection{Double Not}
\label{res:bug:DoubleNot}
The first bug we discovered was our "double not"-bug, a bug where we ask CPMpy to solve the constraints "X==3 and not(not(X==3))". Clearly this solution is trivial, set variable 'X' equal to 3 and the problem would be satisfied. However not all CPMpy solver did agreed with this, both OR-Tools and Gurobi said that this problem was unsatisfiable. 

This was due to a process within CPMpy responsible for creating a flat normal form. Not all solver used by CPMpy allow an arbitrary nesting of constraints as described by the documentation of CPMpy\footnote{\url{https://cpmpy.readthedocs.io/en/latest/behind_the_scenes.html}}. It is for that reason that CPMpy flattens the constraints to what they call 'flat normal forms' as the similar definition of SAT. But, with a disclaimer that this definition does not formally exists for CP languages to their knowledge, a statement which we agree with. With this flattened form CPMpy can directly call the solvers or do the last changes needed for the specific solver via the solver interface on the flattened constraints to then send it to the respective solver \cite{CPMpyGithub}. 

Those not's get translated to two comparison with a zero "== 0", but in the normalizing process that a comparison withing a comparison was not handled correctly. Causing a disappearance of a single "not", which in turn resulted the original constraint converted to "X==3 and not(X==3)". When this gets sent to OR-Tools or Gurobi they correctly say that the problem is unsatisfiable. The other solvers, mainly MiniZinc's subsolvers, were not affected by this bug due to not using this normalizing process. Although, this normalizing process was subjected to unit tests, these tests contained an incorrect output causing the bug to remain hidden. This bug was only caught using CTORM, due to its frequent use of adding not's and and's. But we believe it could have been caught in the metamorphic testing if we had thought of adding a relevant metamorphic transformation or in the differential testing if a seed had a double "not" in its constraints. 

A showcase of this double not bug can be seen in listing \ref{lst:Bug:DoubleNot}, where a variable 'X' is created on line 3 with a lower bound (lb) of zero and an upper bound of 9 (ub). Then, add the constraint "X == 3" to a created model on line 5 with the "+=" and the same constraint with a double negation on the next line. Remember that CPMpy uses '$\sim$' as a negation. We then see the different solvers solve the same model with a different exit status, unsatisfiable for OR-Tools and Gurobi and feasible for a MiniZinc subsolver. Feasible is a differentiation made by CPMpy within satisfiable with the other option being optimal both explain themselves. This bug report can be found in the GitHub repository issue number \href{https://github.com/CPMpy/cpmpy/issues/142}{142}.

\label{lst:Bug:DoubleNot}
\begin{lstlisting}[language=python, caption={The "double not"-bug.}]
	from cpmpy import *
	
	X = intvar(lb=0, ub=9)
	m = Model()
	m += X == 3
	m += ~(~(X == 3)) # double negation 
	
	m.solve(solver="gurobi")
	print(m.status().exitstatus.name) # UNSATISFIABLE
	
	m.solve(solver="ortools")
	print(m.status().exitstatus.name) # UNSATISFIABLE
	
	m.solve(solver="minizinc:chuffed")
	print(m.status().exitstatus.name) # FEASIBLE	
\end{lstlisting}

\subsection{Negation of Global functions}
\label{res:bug:NegatedGlobal}
A second bug related to the use of not's were the crashes of the negated global functions. Where negations of global functions like "not(AllDifferent(argList))" would crash with a maximal recursion depth. As the normalizing of a negated global function would be handled with adding a "== 0" to it, instead of decomposing the global function and negating that. The action of adding a "== 0" did not change the constraint and on the next normalizing of the left part of the comparison "global function == 0", the same would happen. The solution was, as mentioned to decompose the global function, which was suggested in the comments together with a commented out throw of an not-implemented error. The entire function was labeled as work in progress, but the CPMpy-team expected it to work for this use case as they used it in their reification process. The reason it worked in this process was due to a shortcut not being taken, which the negation of global functions did do, therefore exposing the bug only in the latter case.

It was again due to the normalizing not being used for MiniZinc's subsolvers that the bug only occurred when using OR-Tools and Gurobi as solvers. And this bug was quickly found by CTORM since it uses a significant number of not's. Due to the metamorphic relation of adding "!= 0" after some constraints the metamorphic tests did find it as well. However, it did not get found by our differential tests simply because no examples negated their global functions, as such a negation is rarely useful. 


A showcase of this bug can be seen in listing \ref{lst:Bug:NotGlobal}, where a variable "pos" is created on line 3 with a shape of 3 meaning that "pos" will be an array of length 3. After creating an empty model a negation of a global function "AllDifferent" is added to a created model on line 5. With this we ask to find not all different values of the array "pos", which will be satisfies as long as one of the elements withing the array has the same value as one of the other elements in the array. For example, array "[1 2 2]" would satisfy. Subsequently, a MiniZinc subsolver is used to solve the problem which returns feasible on respectively line 7 and 8. However, when asking the same for solver Gurobi it will crash, analog with the next line if the previous one would not have crashed the program. This bug report can be found in the GitHub repository issue number  \href{https://github.com/CPMpy/cpmpy/issues/143}{143}.

\label{lst:Bug:NotGlobal}
\begin{lstlisting}[language=python, caption={The "negation of global functions"-bug.}]
	from cpmpy import *

	pos = intvar(lb=0, ub=5, shape=3)
	m = Model()
	m += ~AllDifferent(pos)
	
	m.solve("minizinc:chuffed")
	print(m.status().exitstatus.name) # FEASIBLE

	m.solve("gurobi") # crash
	m.solve("ortools") # would crash as well
\end{lstlisting}

\subsection{Power function of Gurobi}
\label{res:bug:Power}
Now that we have seen two bugs in the normalization part of CPMpy, which both fit in the transformation component of CPMpy. It is time to look at the solver interface and some bugs we found there. A first one was a bug where the solver Gurobi would crash if we gave a base variable in the power function which had a negative lower bound. A lower bound meaning that the variable was not permitted lower that that bound. All other solver would be able to solve "pow(X, 2) == 9" with the variable 'X' defined with a lower bound of -5 and a higher bound of 5. But Gurobi did (and still does) not allow this throwing an error as a result as can be seen in listing \ref{lst:Bug:PowGurobi}.

This bug was found with the CTORM implementation because there was a seed file which contained this power function with a negative lower bound in the base. However, the solver used to solve this problem was not Gurobi meaning that the bug was not discovered when written. The original example can be found at the CPMpy repository's csplib examples\footnote{\url{https://github.com/CPMpy/cpmpy/blob/b60310d7962bc7631bcf0b9024140e47c1fb302e/examples/csplib/prob005_auto_correlation.py}}. We do not think that CTORM would have found it, if it was not in the seed file to begin with. This because CTORM does not create new variables nor modifies any bounds of variables. The bug was also not found using the metamorphic tests, but we do believe that if we had written a metamorphic relation for the power function or one where we changed some bounds of variables that we could have found this bug. And since the problem was already in the seed file to begin with the differential testing did find the bug and logged it. This bug report can be found in the GitHub repository issue number \href{https://github.com/CPMpy/cpmpy/issues/149}{149}.

\label{lst:Bug:PowGurobi}
\begin{lstlisting}[language=python, caption={The "power function of Gurobi"-bug.}]
	from cpmpy import *

	m = Model()
	X = intvar(lb=-5, ub=5)
	m += pow(X, 2) == 9

	m.solve(solver="gurobi") # GurobiError
\end{lstlisting}



\subsection{Wrong bound value Error}
\label{res:bug:WrongBounds}
A second bug we found in the solver interface was a missing check on the variable type, this time not in Gurobi but in the PySAT implementation. When asking for a check if a sum of booleans matches a specific variable and that variable happens to be an integer instead of a boolean naturally the SAT solver, which only support boolean satisfiability problems, will complain. In this specific case it was a follow-up function still within CPMpy that crashed. This with wrong bounds since it expected a bound of only two possibilities, a boolean, but got a larger bounds. On all other places we could find check with an error that would be thrown. But on this spot it was missed, which was quickly patched after reporting it.

Although, CTORM was run with PySAT's subsolvers, it did not find this bug simply due to a check of (un)satisfiability of the original problem at the start of the program. This check would often crash with PySAT's subsolvers since almost all seeds were written with a CP solver in mind. The technique would assume that the original seed was faulty to start with and continue with another solver or another seed. The same happened with the metamorphic tests where we needed to know the (un)satisfiability of the model before the changes, where it would crash again on the original model. Since those crashes were not logged in both techniques, we did not find it with this techniques. The bug did get discovered with the differential tester where each crash did get logged on top of all differences. This bug report can be found in the GitHub repository issue number \href{https://github.com/CPMpy/cpmpy/issues/150}{150}.

%\subsection{Circuit of one} % remove this bug from the txt ?
%\label{res:bug:Circuit}
%On top of bugs related to specific solver or a group of solvers treated differently like the first two bugs. We also found bugs that would be thrown unrelated to which solver was used to solve the problem. Which brings us to the bug where we discovered that creating a global function, namely circuit, with only one entry would it crash with a not subscriptable error. This bug can be seen as an user caused error and be dismissed, but the CPMpy-team agreed saw it the same way. They marked it as a bug and added a small check.
%
%\todo{which fuzzers found it?}
%
%This bug report can be found in the GitHub repository issue number \href{https://github.com/CPMpy/cpmpy/issues/157}{157}.

\subsection{Naming variables}
\label{res:bug:Naming+andImport}
Now that we have seen bugs occur in both the transformations and the solver interface let us look at a bug we have found in the model component of CPMpy. 

CPMpy has multiple features like importing, exporting models, adding names variables (not to be confused with the local variables as seen on line 12 in listing \ref{lst:SendMoreMoneyCPMpy}) and more. The adding of the name is to make sure that after an export and import the given variable names are still remembered among other reasons, like to be able to give the solver the variable names. When the programmer does not give a name to a variable as we did on line 12 in listing \ref{lst:SendMoreMoneyCPMpy} with the missing "name=''" attribute in the "intvar" function. Then CPMpy adds a name to the variable without telling the programmer. these variables start with "BV" for boolean variables and "IV" with integer variables and each get appended with their respective incrementing number to prevent similar names. This because reusing variable names is dangerous when the solvers use this name to differentiate variables from each other.
%\todo{is dit slecht uitgelegd?}

This brings us to the bug; it occurs when importing a model with automatic naming where those counters did not get updated. Meaning that when a new variable was created with automatic naming it would have an overlapping variable name with a variable that was previously imported. When a solver was then called it would treat both variables as the same resulting in potential wrongly unsatisfiable solutions. The use case is a bit farther from the normal use case a programmer would go through. Nevertheless, this was not considered a misuse of CPMpy according to the developers and at the time of writing a pull request got proposed in which the import function got extended to check the highest occurrence of the boolean and integer counter. This highest occurrence will then be used for the counters of new variables.

A similar and almost related bug is in the naming of variables, when creating them starting with strange symbols like '+', '\%' or others some solvers would crash. Most solvers would happily solve with these names, but all MiniZinc's subsolvers crashed with a syntax error when handling the input. This due to transformation of our model to the text-based FlatZinc for the subsolver, it can no longer differentiate between the variable name and the code. It therefore crashed when seeing anything that could be interpreted differently than a variable name. MiniZinc does state that identifiers are not allowed to contain special characters, which other solvers and CPMpy do allow. A solution is still being discussed at the time of writing.


In listing \ref{lst:Bug:+} this bug of strange symbols is showcased. With a variable 'i' being declared on line 3 with a lower and higher bound respectively 0 and 5. To then define a name manually and name it '+', this in contrast with the previous listings where CPMpy used automatically naming of the variables.  Due to this strange naming of variables MiniZinc will crash with a syntax error on line 10 while it solved the constraint fine on line 7.

\label{lst:Bug:+}
\begin{lstlisting}[language=python, caption={A bug showcasing that the naming of CPMpy is looser then MiniZinc.}]
	from cpmpy import *
	
	i = intvar(lb=0, ub=5, name="+")
	m = Model()
	m += i > 0
	
	m.solve(solver="ortools")
	print(m.sstatus().exitstatus.name) # OPTIMAL
	
	m.solve(solver="minizinc:chuffed") # crash by syntax error
\end{lstlisting}

Both of these bugs were not caught by CTORM nor the differential testing, since they do not create new variables. But did get caught by the metamorphic tests, the first bug was caught because we imported a seed file where automatic naming was done after which we created a variable too with this process, resulting in the the bug. the second one is a bit more embarrassing to write down, as we created a bug in the metamorphic tester which resulted in the (unintentionally) creation of variables starting with a '+'. Our own bug caused us to find a bug in CPMpy. We still label it a caught bug because the automatic bug catcher did find it, although only by a fault we made. This bug reports can be found in the GitHub repository issue number \href{https://github.com/CPMpy/cpmpy/issues/158}{158} and \href{https://github.com/CPMpy/cpmpy/issues/162}{162} respectively.


\subsection{MiniZinc returning zero}
\label{res:bug:MinizincZero}
Our last bug we will discuss in detail was a bug we found with a solver themselves, namely with some MiniZinc subsolver. While solving certain problems with MiniZinc's subsolvers Gecode and others it would sometimes crash with the error that it stopped without output. After reporting it turned out to be a known bug in the MiniZinc Python repository for Windows operating systems and was fixable with setting some path variables correctly. Which CPMpy may solve by adding a warning when this happens or by documenting it.
Given that this problem is an installation problem, all techniques were able to find the bug. This bug report can be found in the GitHub repository issue number \href{https://github.com/CPMpy/cpmpy/issues/156}{156}.

%\subsection{Unsatisfiable Gurobi} % boring bug
%\label{res:bug:UnsatGurobu}
%Our last bug we will discuss is a bug that was found by all three techniques
%
%
%This bug report can be found in the GitHub repository issue number \href{https://github.com/CPMpy/cpmpy/issues/168}{168}.


%\subsection{Solver lookup bug} % boring bug
%\label{res:bug:solverLookup}
%solverlookup() was during development of the tester = manual testing, 

\todo{bug \href{https://github.com/CPMpy/cpmpy/issues/163}{163} would be a fun explanation for the thesis, but lacks a bug. atm it's a note in \ref{impl:Meta}}


\section{Classifications}
Now that we have seen in depth explanations of some bugs let us give an overview of all found bugs by classifying them based on place, type of the bugs, which solver caused the bugs and which technique found the bugs. The bug number refers to the issue number on GitHub and is a hyperlink to that bug, the second column is a short description of the bug and then the table specific classification follows.

% added above the text that refference it, to prevent tables being back to back
\begin{table}[]
	\centering
	\caption{Table discussing in which CPMpy component the bug was found. With 4 bugs in the model, 7 bugs in the transformations, 7 bugs in the solver interface and one a solver were found.}
	\label{tab:bug:placeComponent}
	\begin{tabular}{lll}
		\hline
		BugNr & Bug description                                         & Place of the bug \\ \toprule
		\href{https://github.com/CPMpy/cpmpy/issues/142}{142} & double not gives unsat                            & Transformations \\
		\href{https://github.com/CPMpy/cpmpy/issues/143}{143} & negating global functions crashes                 & Transformations \\
		\href{https://github.com/CPMpy/cpmpy/issues/145}{145} & solvers lookup crashes                            & Model            \\
		\href{https://github.com/CPMpy/cpmpy/issues/149}{149} & power function with negative lower bound crashes  & Solver interface \\
		\href{https://github.com/CPMpy/cpmpy/issues/150}{150} & wrong bound causes a crash                        & Solver interface \\
		\href{https://github.com/CPMpy/cpmpy/issues/152}{152} & boolean variable does not support implies         & Model            \\
		\href{https://github.com/CPMpy/cpmpy/issues/153}{153} & Gurobi does not run and gave the wrong nr of sol  & Solver interface \\
		\href{https://github.com/CPMpy/cpmpy/issues/154}{154} & JSON Decoder error                                & Solver interface \\
		\href{https://github.com/CPMpy/cpmpy/issues/155}{155} & list has no shape                                 & Solver interface \\
		\href{https://github.com/CPMpy/cpmpy/issues/156}{156} & MiniZinc returns zero causes a crash              & Solver           \\
		\href{https://github.com/CPMpy/cpmpy/issues/157}{157} & circuit of one element crashes                    & Transformations  \\
		\href{https://github.com/CPMpy/cpmpy/issues/158}{158} & identical variable name can cause wrongly unsat   & Model            \\
		\href{https://github.com/CPMpy/cpmpy/issues/159}{159} & unhandled Gurobi exit status 9                    & Solver interface \\
		\href{https://github.com/CPMpy/cpmpy/issues/161}{161} & two separate references for the same variable     & Model            \\
		\href{https://github.com/CPMpy/cpmpy/issues/162}{162} & CPMpy is looser with variable names than MiniZinc & Solver interface \\
		%\href{https://github.com/CPMpy/cpmpy/issues/163}{163} & cyclic expression tree got generated              & Model            \\
		\href{https://github.com/CPMpy/cpmpy/issues/164}{164} & malloc() failure due to unset bounds              & Transformations  \\
		\href{https://github.com/CPMpy/cpmpy/issues/165}{165} & memory violation segmentation fault               & Transformations  \\
		\href{https://github.com/CPMpy/cpmpy/issues/168}{168} & unsatisfiable Gurobi                              & Transformations  \\
		\href{https://github.com/CPMpy/cpmpy/issues/170}{170} & unsatisfiable due to flattening                   & Transformations  \\ \bottomrule        
	\end{tabular}
\end{table}

%\subsection{Place of the bug}
As can be seen in table \ref{tab:bug:placeComponent} the cause of which component failed is well spread out withing CPMpy. 
With 4 bugs in the model, 7 bugs in the general transformations, 7 bugs in the solver interface and one a solver related bug. The one and only solver related bug we found was the one we discussed in subsection \ref{res:bug:MinizincZero}, which was already known by MiniZinc. We would have hoped to find more bugs in the solvers themselves and it was our aim with this thesis. But either these techniques are not sufficient or most bugs are already found before release or are already reported and solved.
If we look at the reported issues withing the GitHub repository of Google's OR-Tools, we find no significant bugs towards the (un)satisfiability or any wrong output by the solver. Which makes us speculate that Google does extensive testing on that front or even have used the techniques used in this thesis. This last one is likely as Google created multiple own fuzzers, which we discussed in subsection  \ref{fuzzing:OtherFuzzers}. Extensive testing is most likely also done by other solvers since they would probably lose reputation if their solver would be proven to not produce the correct result.

%\subsection{What did the bug cause}
Like the authors of STORM, we focused with our techniques on the critical faults, this being the wrongly satisfiable, the wrongly unsatisfiable and the wrong number of solutions. Since these critical bugs are harder to detect for the final user than a crash, timeout or other bug. Out of the 19 bugs found 6 of them fall in our category of critical while the other 13 where all crashes as can be seen in table \ref{tab:bug:fault}. Most of those 6 critical bugs were situations where the solver wrongly outputted that a solution was unsatisfiable and there was only one bug where we could find both a wrongly satisfiable and wrongly unsatisfiable solution.

\begin{table}[]
	\caption{Table discussing what type of fault was caused by which the bugs.}
	\label{tab:bug:fault}
	\centering
	\begin{tabular}{lll}
		\hline
		BugNr & Bug description                                           & Type of fault   \\ \toprule
		\href{https://github.com/CPMpy/cpmpy/issues/142}{142} & double not gives unsat                            & wrongly unsat   \\
		\href{https://github.com/CPMpy/cpmpy/issues/143}{143} & negating global functions crashes                 & crash           \\
		\href{https://github.com/CPMpy/cpmpy/issues/145}{145} & solvers lookup crashes                            & crash           \\
		\href{https://github.com/CPMpy/cpmpy/issues/149}{149} & power function with negative lower bound crashes  & crash           \\
		\href{https://github.com/CPMpy/cpmpy/issues/150}{150} & wrong bound causes a crash                  & crash           \\
		\href{https://github.com/CPMpy/cpmpy/issues/152}{152} & boolean variable does not support implies         & crash           \\
		\href{https://github.com/CPMpy/cpmpy/issues/153}{153} & Gurobi does not run and gave the wrong Nr of sol  & wrong Nr of sol \\
		\href{https://github.com/CPMpy/cpmpy/issues/154}{154} & JSON Decoder error                                & crash           \\
		\href{https://github.com/CPMpy/cpmpy/issues/155}{155} & list has no shape                                 & crash           \\
		\href{https://github.com/CPMpy/cpmpy/issues/156}{156} & MiniZinc returns zero causes a crash              & crash           \\
		\href{https://github.com/CPMpy/cpmpy/issues/157}{157} & circuit of one element crashes                    & crash           \\
		\href{https://github.com/CPMpy/cpmpy/issues/158}{158} & identical variable name can cause wrongly unsat   & wrongly unsat   \\
		\href{https://github.com/CPMpy/cpmpy/issues/159}{159} & unhandled Gurobi exit status 9                    & crash           \\
		\href{https://github.com/CPMpy/cpmpy/issues/161}{161} & two separate references for the same variable     & wrongly unsat   \\
		\href{https://github.com/CPMpy/cpmpy/issues/162}{162} & CPMpy is looser with variable names than MiniZinc & crash           \\
		%\href{https://github.com/CPMpy/cpmpy/issues/163}{163} & cyclic expression tree got generated               & crash           \\
		\href{https://github.com/CPMpy/cpmpy/issues/164}{164} & malloc() failure due to unset bounds              & crash           \\
		\href{https://github.com/CPMpy/cpmpy/issues/165}{165} & memory violation segmentation fault               & crash           \\
		\href{https://github.com/CPMpy/cpmpy/issues/168}{168} & wrongly unsatisfiable Gurobi                      & wrongly unsat   \\
		\href{https://github.com/CPMpy/cpmpy/issues/170}{170} & wrongly (un)satisfiable due to flattening         & wrongly (un)sat \\ \bottomrule
	\end{tabular}
\end{table}

%\subsection{Which solver was responcible for the bug}
When looking at table \ref{tab:bug:Solver} we can see which bug was caused by which solver or if it was a solver independent bug. Were we see 5 bugs unrelated to any solver, that OR-tools only occurred together with Gurobi and that OR-Tools and Gurobi didn't share any bugs found with MiniZinc or PySAT. Meanly because OR-Tools and Gurobi share more transformation code than any other solver. We also see that Gurobi occurs the most among out our bugs, this often due to edge cases on Gurobi's implementation. Although, it could be coincidental that we found more Gurobi bug than any other we speculate that the software being proprietary makes it less clear-cut to be implemented in CPMpy.

\begin{table}[]
	\centering
	\caption{Table discussing which bug was caused by which solver or if it was a solver independent bug.}
	\label{tab:bug:Solver}
	\begin{tabular}{lll}
		\hline
		BugNr & Bug description                                           & Which solver caused it?\\ \toprule
		\href{https://github.com/CPMpy/cpmpy/issues/142}{142} & double not gives unsat                            & OR-Tools and Gurobi          \\
		\href{https://github.com/CPMpy/cpmpy/issues/143}{143} & negating global functions crashes                 & OR-Tools and Gurobi          \\
		\href{https://github.com/CPMpy/cpmpy/issues/145}{145} & solvers lookup crashes                            & solver independent           \\
		\href{https://github.com/CPMpy/cpmpy/issues/149}{149} & power function with negative lower bound crashes  & Gurobi                       \\
		\href{https://github.com/CPMpy/cpmpy/issues/150}{150} & wrong bound causes a crash                  & all PySAT subsolvers         \\
		\href{https://github.com/CPMpy/cpmpy/issues/152}{152} & boolean variable does not support implies         & solver independent           \\
		\href{https://github.com/CPMpy/cpmpy/issues/153}{153} & Gurobi does not run and gave the wrong nr of sol  & Gurobi                       \\
		\href{https://github.com/CPMpy/cpmpy/issues/154}{154} & JSON Decoder error                                & MiniZinc's subsolver osicbc  \\
		\href{https://github.com/CPMpy/cpmpy/issues/155}{155} & list has no shape                                 & Gurobi                       \\
		\href{https://github.com/CPMpy/cpmpy/issues/156}{156} & MiniZinc returns zero causes a crash              & multiple MiniZinc subsolvers \\
		\href{https://github.com/CPMpy/cpmpy/issues/157}{157} & circuit of one element crashes                    & solver independent           \\
		\href{https://github.com/CPMpy/cpmpy/issues/158}{158} & identical variable name can cause wrongly unsat   & solver independent           \\
		\href{https://github.com/CPMpy/cpmpy/issues/159}{159} & unhandled Gurobi exit status 9                    & Gurobi                       \\
		\href{https://github.com/CPMpy/cpmpy/issues/161}{161} & two separate references for the same variable     & solver independent           \\
		\href{https://github.com/CPMpy/cpmpy/issues/162}{162} & CPMpy is looser with variable names than MiniZinc & all MiniZinc subsolvers      \\
		%\href{https://github.com/CPMpy/cpmpy/issues/163}{163} & cyclic expression tree got generated               & solver independent           \\
		\href{https://github.com/CPMpy/cpmpy/issues/164}{164} & malloc() failure due to unset bounds              & multiple MiniZinc subsolvers \\
		\href{https://github.com/CPMpy/cpmpy/issues/165}{165} & memory violation segmentation fault               & multiple MiniZinc subsolvers \\
		\href{https://github.com/CPMpy/cpmpy/issues/168}{168} & wrongly unsatisfiable Gurobi                      & Gurobi                       \\
		\href{https://github.com/CPMpy/cpmpy/issues/170}{170} & wrongly (un)satisfiable due to flattening         & OR-Tools and Gurobi          \\ \bottomrule
	\end{tabular}
\end{table}

%\subsection{Which technique found the bug}
Our last table \ref{tab:bug:Technique} shows the techniques finding which bug. CTORM found 10 bugs, metamorphic testing found the most bugs at 13 and differential testing found 11 out of the19 found bugs. This shows that none of the techniques are perfect on their own, but that metamorphic testing could come close if more work was put in creating metamorphic relations. Although, this does require creativity and manual labor instead of the other automated techniques.

\begin{table}[]
	\centering
	\caption{Table discussing which technique found the bug. CTORM found 10 bugs, metamorphic testing found the most bugs at 13 and differential testing found 11 out of the 19 found bugs.}
	\label{tab:bug:Technique}
	\begin{tabular}{lllll}
		\hline
		BugNr & Bug description                                           & \multicolumn{3}{c}{\centering  Bug found by} \\ \toprule
		\href{https://github.com/CPMpy/cpmpy/issues/142}{142} & double not gives unsat                            & ctorm &       &      \\
		\href{https://github.com/CPMpy/cpmpy/issues/143}{143} & negating global functions crashes                 & ctorm & meta  &      \\
		\href{https://github.com/CPMpy/cpmpy/issues/145}{145} & solvers lookup crashes                            &       &       & diff \\
		\href{https://github.com/CPMpy/cpmpy/issues/149}{149} & power function with negative lower bound crashes  & ctorm &       & diff \\
		\href{https://github.com/CPMpy/cpmpy/issues/150}{150} & wrong bound causes a crash                  &       &       & diff \\
		\href{https://github.com/CPMpy/cpmpy/issues/152}{152} & boolean variable does not support implies         &       & meta  & diff \\
		\href{https://github.com/CPMpy/cpmpy/issues/153}{153} & Gurobi does not run and gave the wrong nr of sol  &       &       & diff \\
		\href{https://github.com/CPMpy/cpmpy/issues/154}{154} & JSON Decoder error                                & ctorm & meta  & diff \\
		\href{https://github.com/CPMpy/cpmpy/issues/155}{155} & list has no shape                                 & ctorm & meta  & diff \\
		\href{https://github.com/CPMpy/cpmpy/issues/156}{156} & MiniZinc returns zero causes a crash              & ctorm & meta  & diff \\
		\href{https://github.com/CPMpy/cpmpy/issues/157}{157} & circuit of one element crashes                    &       & meta  &      \\
		\href{https://github.com/CPMpy/cpmpy/issues/158}{158} & identical variable name can cause wrongly unsat   &       & meta  &      \\
		\href{https://github.com/CPMpy/cpmpy/issues/159}{159} & unhandled Gurobi exit status 9                    & ctorm &       & diff \\
		\href{https://github.com/CPMpy/cpmpy/issues/161}{161} & two separate references for the same variable     & ctorm & meta  &      \\
		\href{https://github.com/CPMpy/cpmpy/issues/162}{162} & CPMpy is looser with variable names than MiniZinc &       & meta  &      \\
		%\href{https://github.com/CPMpy/cpmpy/issues/163}{163} & cyclic expression tree got generated               &       & meta  &      \\
		\href{https://github.com/CPMpy/cpmpy/issues/164}{164} & malloc() failure due to unset bounds              &       & meta  &      \\
		\href{https://github.com/CPMpy/cpmpy/issues/165}{165} & memory violation segmentation fault               &       & meta  & diff \\
		\href{https://github.com/CPMpy/cpmpy/issues/168}{168} & wrongly unsatisfiable Gurobi                      & ctorm & meta  & diff \\
		\href{https://github.com/CPMpy/cpmpy/issues/170}{170} & wrongly (un)satisfiable due to flattening         & ctorm & meta  &      \\ \bottomrule
	\end{tabular}
\end{table}


\section{Reception to the bugs} 
As mentioned in section \ref{fuzzing:OpinionsAgainstFuzzing} there are multiple views on automated bug catching. We could have thrown all our found bugs on the issue page of GitHub without further context. Which would have meant more work for the CPMpy-team and then we could perhaps have seen some negative opinions on fuzzing. 
Although, we submitted 22 bugs or questions withing a time span of 2 weeks with deobfuscating the inputs and some explanation of what happened to get the bug, we saw a grateful welcome. Similar to what was described in the second part of section \ref{fuzzing:OpinionsAgainstFuzzing}. 
For example, the "double not"-bug\footnote{\url{https://github.com/CPMpy/cpmpy/issues/142}} was called a serious bug and a great find and 
the "negation of global functions"-bug\footnote{\url{https://github.com/CPMpy/cpmpy/issues/143}} was described as an unexpected bug and another great find.
%\todo{find out if this is useful addition of the thesis?}

%\section{unsat}
%due to the way of importing the file a lot of edge problems become sat or unsat



\section{Conclusion}
\label{res:conclusion}
In this chapter we have seen that the techniques frequently output already found bugs, since none of the techniques have knowledge of already found bugs. But after filtering previously found bugs the techniques perform well and found 19 bugs in CPMpy. We have seen some bugs in detail with most of them already fixed, due to being easily fixable. We have seen that most bugs were related to the CPMpy code and only one responsible by an external solver. We suspect this lack of external solver bugs to be caused by well written and tested solvers. We have shown that we found crashes up to critical bugs like wrongly (un)satisfiable solutions and wrong number of solutions with over a variety of solvers. Of all the techniques used, metamorphic testing came just above the other two techniques with 13 found bugs while CTORM found 10 and the differential testing found 11 bugs out of 19. And finally, we noted the grateful welcome of bugs by the CPMpy-team.

%%% Local Variables: 
%%% mode: latex
%%% TeX-master: "thesis"
%%% End: 
 % conclustion
%\chapter{creation of fuzzer}
\label{cha:x}
\todo{intro ch x}

% own results, multiple runs due toe randomness, how measured
% comparision with other fuzzers

\section{creation of sat and unsat formulas}
see paper 43 p 4
Semantic Fusion
\cite{43YinYang}
een ref to chapter 2

seed files came from the CPMpy repository the main branch and the csplib branch aswell from Hakank's repository downloaded on Tuesday 27/09/2022
Storm downloaded on Tuesday 27/09/2022 from https://github.com/Practical-Formal-Methods/storm 
https://github.com/Practical-Formal-Methods/storm/commit/55d091624523a0544112ffc339fe81103b3daa2b



Storm take in SMT-lib seed files, we can convert them to minizinc and then using fzn2smt but It hasn't been maintained in over a decenta and doing multiple convertions only back could be tricky and would introduce multiple layers which can introduce bugs a normal user would ever see
Therefore we will be refactoring STORM to fit our CPMpy


























%
%\section{Figures}
%Figures are used to add illustrations to the text. The \fref{fig:logo} shows
%the KU~Leuven logo as an illustration.
%\begin{figure}
%	\centering
%	\includegraphics{logokul}
%	\caption{The KU~Leuven logo.}
%	\label{fig:logo}
%\end{figure}
%
%\section{Tables}
%Tables are used to present data neatly arranged. A table is normally
%not a spreadsheet! Compare \tref{tab:wrong} en \tref{tab:ok}: which table do
%you prefer?
%
%\begin{table}
%	\centering
%	\begin{tabular}{||l|lr||} \hline
%		gnats     & gram      & \$13.65 \\ \cline{2-3}
%		& each      & .01 \\ \hline
%		gnu       & stuffed   & 92.50 \\ \cline{1-1} \cline{3-3}
%		emu       &           & 33.33 \\ \hline
%		armadillo & frozen    & 8.99 \\ \hline
%	\end{tabular}
%	\caption{A table with the wrong layout.}
%	\label{tab:wrong}
%\end{table}
%
%\begin{table}
%	\centering
%	\begin{tabular}{@{}llr@{}} \toprule
%		\multicolumn{2}{c}{Item} \\ \cmidrule(r){1-2}
%		Animal    & Description & Price (\$)\\ \midrule
%		Gnat      & per gram    & 13.65 \\
%		& each        & 0.01 \\
%		Gnu       & stuffed     & 92.50 \\
%		Emu       & stuffed     & 33.33 \\
%		Armadillo & frozen      & 8.99 \\ \bottomrule
%	\end{tabular}
%	\caption{A table with the correct layout.}
%	\label{tab:ok}
%\end{table}
%

\section{Conclusion}
The final section of the chapter gives an overview of the important results
of this chapter. This implies that the introductory chapter and the
concluding chapter don't need a conclusion.

%%% Local Variables: 
%%% mode: latex
%%% TeX-master: "thesis"
%%% End: 
 % futher work ?

% Indien er bijlagen zijn:
%\appendixpage*          % indien gewenst
%\appendix
%\chapter{Manual for setting up the fuzzer}
\label{app:A}


%%% Local Variables: 
%%% mode: latex
%%% TeX-master: "thesis"
%%% End: 


\backmatter
% Na de bijlagen plaatst men nog de bibliografie.
% Je kan de  standaard "abbrv" bibliografiestijl vervangen door een andere.
%\bibliographystyle{abbrv}
%\bibliography{references}
\nocite{MasterproefRubenKindt}
\printbibliography
\end{document}

%%% Local Variables: 
%%% mode: latex
%%% TeX-master: t
%%% End: 
